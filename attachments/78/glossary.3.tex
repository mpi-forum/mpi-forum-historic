% glossary.tex
%
%
%
% Version as of Sept-02-2011
% Edited by Terry Jones
%
%
%
%%\status{pre First Reading}


\chapter{Glossary}
\label{glossary}

\textbf{begin discussion ~\textgreater{}---------------------} \\*
\\*
The following Annex is provided to promote precise usage of MPI concepts and specialized terms,
to provide a convenient mechanism to learn  \MPI/ nomenclature (especially for new users),
and to help ensure consistency in use of functions, operations, etc.

\vskip 0.2truein
\begin{center}
\begin{tabular}{l c |}
{\bf Glossary Inclusion Criteria} \\ 
\hline
{\small The criteria for Glossary inclusion are} \\
\hline
\end{tabular}
\begin{tabular}{l l |}
\hline
The term is associated with version \MPIIII/  ballot xx. \\[0.2cm]
The term is used at least once in the document. \\[0.2cm]
The term is well defined and has a clear context. \\[0.2cm]
Terms that are used in different ways in the MPI-3 spec will have multiple entries. \\[0.2cm]
Restatement of definitions found within the specification are allowed. \\[0.2cm]
Motivation is included within the definition when helpful. \\[0.2cm]
General terms used in accordance with the standard definition are included when \\
deemed important to \MPIIII/ (e.g., thread safe).\\[0.2cm]
\hline \\*
\end{tabular}
\end{center}
\textbf{---------------------~\textless{} end discussion}  \\*



\begin{itemize}

%%%%%%%%%%%%%%%%%%%%%%%%%%%%%%%%%%%%%%%%%%%%%%%%%%%%%%%%%%%%%%%%%%%%%%
\label {glossary:absolute_address}
\item  ~\hypertarget{glossary:absolute_address}{\textbf{absolute address}} \\*
Displacements relative to  \const{MPI\_BOTTOM}, 
the start of the address space. Note that  \const{MPI\_BOTTOM}
may be viewed as a "zero address" but need not be zero.
Refer to Section~\ref{subsec:named-constants} on page~\pageref{subsec:named-constants}.

%%%%%%%%%%%%%%%%%%%%%%%%%%%%%%%%%%%%%%%%%%%%%%%%%%%%%%%%%%%%%%%%%%%%%%
\label{glossary:access_epoch}
\item  ~\hypertarget{glossary:access_epoch}{\textbf{access epoch}} \\*
The period during which a target window can be accessed by \RMA/
operations.
(See also ~\hyperlink{glossary:exposure_epoch}{exposure epoch} and
 ~\hyperlink{glossary:RMA}{RMA}.)
Refer to Section~\ref{sec:1sided-sync} on page~\pageref{sec:1sided-sync}.

%%%%%%%%%%%%%%%%%%%%%%%%%%%%%%%%%%%%%%%%%%%%%%%%%%%%%%%%%%%%%%%%%%%%%%
\label{glossary:active}
\item  ~\hypertarget{glossary:active}{\textbf{active}} \\*
The property of a process within a parallel procedure that
belongs to a group that may collectively execute the procedure, and
some member of that group is currently executing the procedure code.
If a parallel procedure is active in a process, then this process may
be receiving messages pertaining to this procedure, even if it
does not currently execute the code of this procedure.
Refer to Section~\ref{sec:formalizing} on page~\pageref{sec:formalizing}.

%%%%%%%%%%%%%%%%%%%%%%%%%%%%%%%%%%%%%%%%%%%%%%%%%%%%%%%%%%%%%%%%%%%%%%
\label{glossary:active_target_communication}
\item  ~\hypertarget{glossary:active_target_communication}{\textbf{active target communication}} \\*
An \RMA/ communication where data is moved from the memory of one
process to the memory of another, and both are explicitly involved in the
%%communication.  
synhcronization. This communication pattern is similar to message
passing, except that all the data transfer arguments are provided by
one process, and the second process only participates in the synchronization.
(See also ~\hyperlink{glossary:passive_target_communication}{passive target communication}  and
 ~\hyperlink{glossary:RMA}{RMA}.).)
Refer to Section~\ref{sec:1sided-sync} on page~\pageref{sec:1sided-sync}.

%%%%%%%%%%%%%%%%%%%%%%%%%%%%%%%%%%%%%%%%%%%%%%%%%%%%%%%%%%%%%%%%%%%%%%
\label{glossary:associative}
\item  ~\hypertarget{glossary:associative}{\textbf{associative}} \\*
The property of a collective reduction operation that the order in which the operations are 
performed does not matter as long as the sequence of the operands is not changed. 
Binary operation $ \otimes $ is associative iff:
\begin{eqnarray}
( A \otimes B )  \otimes C = A   \otimes ( B   \otimes C ) 
\end{eqnarray}
Reduction operation properties apply to \func{MPI\_OP\_CREATE} argument MPI\_Op~*op.
(See also ~\hyperlink{glossary:commutative}{commutative}.)
Refer to Section~\ref{subsec:coll-user-ops} on page~\pageref{subsec:coll-user-ops}.

%%%%%%%%%%%%%%%%%%%%%%%%%%%%%%%%%%%%%%%%%%%%%%%%%%%%%%%%%%%%%%%%%%%%%%
\label{glossary:attributes}
\item  ~\hypertarget{glossary:attributes}{\textbf{attributes}} \\*
Arbitrary pieces of information cached on three kinds of \MPI/ objects:
communicators, windows and datatypes.
(See also ~\hyperlink{glossary:caching}{caching}.)
Refer to Section~\ref{sec:caching} on page~\pageref{sec:caching}.

%%%%%%%%%%%%%%%%%%%%%%%%%%%%%%%%%%%%%%%%%%%%%%%%%%%%%%%%%%%%%%%%%%%%%%
\label{glossary:basic_datatype}
\item  ~\hypertarget{glossary:basic_datatype}{\textbf{basic datatype}} \\*
A basic datatype is a named predefined datatype, which corresponds to
the basic datatypes of the host language. Examples include
\type{MPI\_INTEGER},
\type{MPI\_REAL},
\type{MPI\_DOUBLE\_PRECISION},
\type{MPI\_COMPLEX},
\type{MPI\_LOGICAL}, and
\type{MPI\_CHARACTER}.
(See also ~\hyperlink{glossary:derived_datatype}{derived datatype} .)
Refer to Section~\ref{terms:semantic} on page~\pageref{terms:semantic}.

%%%%%%%%%%%%%%%%%%%%%%%%%%%%%%%%%%%%%%%%%%%%%%%%%%%%%%%%%%%%%%%%%%%%%%
\label{glossary:blocking}
\item  ~\hypertarget{glossary:blocking}{\textbf{blocking}} \\*
A procedure is blocking if return from the procedure indicates the user
is allowed to reuse resources specified in the call.
Refer to Section ~\ref{terms:semantic} on page~\pageref{terms:semantic},
and ~\ref{sec:pt2pt-modes} on page~\pageref{sec:pt2pt-modes}
and ~\ref{sec:pt2pt-basicsendrecv} on page~\pageref{sec:pt2pt-basicsendrecv}.

%%%%%%%%%%%%%%%%%%%%%%%%%%%%%%%%%%%%%%%%%%%%%%%%%%%%%%%%%%%%%%%%%%%%%%
\label{glossary:broadcast}
\item  ~\hypertarget{glossary:broadcast}{\textbf{broadcast}} \\*
A collective operation which communicates data from a specified root process to all processes within the group;
initially just the first process contains the data, but after the
broadcast completes successfully all processes contain it.
Refer to Section~\ref{sec:coll-intro} on page~\pageref{sec:coll-intro}.

%%%%%%%%%%%%%%%%%%%%%%%%%%%%%%%%%%%%%%%%%%%%%%%%%%%%%%%%%%%%%%%%%%%%%%
\label{glossary:buffered_communication_mode}
\item  ~\hypertarget{glossary:buffered_communication_mode}{\textbf{buffered communication mode}} \\*
A point-to-point communication protocol in which the
send operation can be started whether or not a
matching receive has been posted.
It may complete before a matching receive is posted.  However, unlike
the standard send, this operation is {\bf local}, and its
completion does not depend on the occurrence of a matching receive.  Thus, if a
send is executed and no matching receive is posted, then \MPI/ must buffer the
outgoing message, so as to allow the send call to complete.   An error will
occur if there is insufficient buffer space.   The amount of available buffer
space is controlled by the user.
Buffer allocation by the user may be required for the buffered mode to be
effective. 
(See also ~\hyperlink{glossary:standard_communication_mode}{standard communication mode}, 
~\hyperlink{glossary:synchronous_communication_mode}{synchronous communication mode},
~\hyperlink{glossary:ready_communication_mode}{ready communication mode}.)
Refer to Section~\ref{sec:pt2pt-modes} on page~\pageref{sec:pt2pt-modes}.

%%%%%%%%%%%%%%%%%%%%%%%%%%%%%%%%%%%%%%%%%%%%%%%%%%%%%%%%%%%%%%%%%%%%%%
\label{glossary:caching}
\item  ~\hypertarget{glossary:caching}{\textbf{caching}} \\*
To store data in a manner for quick re-use later.
\MPI/ provides a ``caching'' facility that allows an application to
attach arbitrary pieces of information, called {\bf attributes}, to
% communicators.  More precisely, the caching
three kinds of MPI objects: communicators, windows and datatypes.
More precisely, the caching
facility allows any library to do the following:
\begin{itemize}
\item
  pass information between calls by associating it
  with an \MPI/ intra- or in\-ter-\-com\-mun\-i\-ca\-tor, 
window or datatype,
\item quickly retrieve that information, and
\item
 be guaranteed that out-of-date information is never retrieved, even if
 % the communicator is freed and its handle subsequently reused by \MPI/.
 the object is freed and its handle subsequently reused by \MPI/.
\end{itemize}
(See also ~\hyperlink{glossary:attributes}{attributes}.)
Refer to Section~\ref{sec:caching} on page~\pageref{sec:caching}.

%%%%%%%%%%%%%%%%%%%%%%%%%%%%%%%%%%%%%%%%%%%%%%%%%%%%%%%%%%%%%%%%%%%%%%
\label{glossary:cartesian_topology}
\item  ~\hypertarget{glossary:cartesian_topology}{\textbf{cartesian topology}} \\*
A virtual ordering of the processes according to a cartesian grid.
Process coordinates in a Cartesian structure begin their numbering at $0$.
Row-major numbering is always used for the processes in a
Cartesian structure. This means that, for example, the relation
between group rank and coordinates for four processes in
a $(2 \times 2)$ grid is as follows.\\[2.0ex]
\hspace*{\parindent}
\begin{tabular}{ll}
 coord (0,0): & rank 0 \\
 coord (0,1): & rank 1 \\
 coord (1,0): & rank 2 \\
 coord (1,1): & rank 3 \\
\end{tabular} \\
(See also ~\hyperlink{glossary:topology}{topology} and ~\hyperlink{glossary:graph_topology}{graph topology}.)
Refer to Section~\ref{chap:topol} on page~\pageref{chap:topol}.

%%%%%%%%%%%%%%%%%%%%%%%%%%%%%%%%%%%%%%%%%%%%%%%%%%%%%%%%%%%%%%%%%%%%%%
\label{glossary:client}
\item  ~\hypertarget{glossary:client}{\textbf{client}} \\*
The member of a client/server relationship that makes the request. The \emph{client} connects to the server.
 \MPI/ provides a mechanism for two sets of \MPI/  processes that do not share a communicator
to establish communication.
Establishing contact between two groups of processes that do not share an
existing communicator is a collective but asymmetric process.  
(See also ~\hyperlink{glossary:server}{server},  ~\hyperlink{glossary:port_name}{port name},
and ~\hyperlink{glossary:service_name}{service name}). 
Refer to Section~\ref{sec:client-server} on page~\pageref{sec:client-server}.

%%%%%%%%%%%%%%%%%%%%%%%%%%%%%%%%%%%%%%%%%%%%%%%%%%%%%%%%%%%%%%%%%%%%%%
\label{glossary:collective call}
\item  ~\hypertarget{glossary:collective_call}{\textbf{collective call}} \\*
A property of a procedure that all processes in a process group need to invoke the procedure.  A
collective call may or may not be synchronizing.
Collective calls over the same communicator
must be executed in the same order by all members of the process
group.
(See also ~\hyperlink{glossary:collective_communication}{collective communication}.)
Refer to Section~\ref{terms:semantic} on page~\pageref{terms:semantic}.

%%%%%%%%%%%%%%%%%%%%%%%%%%%%%%%%%%%%%%%%%%%%%%%%%%%%%%%%%%%%%%%%%%%%%%
\label{glossary:collective communication}
\item  ~\hypertarget{glossary:collective_communication}{\textbf{collective communication}} \\*
communication that involves a group
or groups of processes. Examples include \mpifunc{MPI\_BARRIER}, \mpifunc{MPI\_BCAST},
\mpifunc{MPI\_GATHER}, \mpifunc{MPI\_GATHERV}, \mpifunc{MPI\_SCATTER}, \mpifunc{MPI\_SCATTERV},
\mpifunc{MPI\_ALLGATHER}, \mpifunc{MPI\_ALLGATHERV}, \mpifunc{MPI\_ALLTOALL}, \mpifunc{MPI\_ALLTOALLV},
\mpifunc{MPI\_ALLTOALLW}, \mpifunc{MPI\_ALLREDUCE}, \mpifunc{MPI\_REDUCE}, \mpifunc{MPI\_IREDUCE\_SCATTER},
\mpifunc{MPI\_SCAN}, and \mpifunc{MPI\_EXSCAN}.
(See also ~\hyperlink{glossary:collective_call}{collective call}.)
Refer to Chapter ~\ref{chap:coll} on page~\pageref{chap:coll}.

%%%%%%%%%%%%%%%%%%%%%%%%%%%%%%%%%%%%%%%%%%%%%%%%%%%%%%%%%%%%%%%%%%%%%%
\label{glossary:committed_datatype}
\item  ~\hypertarget{glossary:committed_datatype}{\textbf{committed datatype}} \\*
The second step in preparing a \emph{derived datatype} for communication (after ``created datatype'').
There is no need to commit basic datatypes. They are ``pre-committed.''
(See also ~\hyperlink{glossary:created_datatype}{created datatype}.)
Refer to Section~\ref{subsec:pt2pt-comfree} on page~\pageref{subsec:pt2pt-comfree}.

%%%%%%%%%%%%%%%%%%%%%%%%%%%%%%%%%%%%%%%%%%%%%%%%%%%%%%%%%%%%%%%%%%%%%%
\label{glossary:communicator}
\item  ~\hypertarget{glossary:communicator}{\textbf{communicator}} \\*
A group with the same communication context.
(See also ~\hyperlink{glossary:communication_context}{communication context}.)
 Refer to Section~\ref{subsec:pt2pt-envelope} on page~\pageref{subsec:pt2pt-envelope} and~\ref{sec:context} on page~\pageref{sec:context}.

%%%%%%%%%%%%%%%%%%%%%%%%%%%%%%%%%%%%%%%%%%%%%%%%%%%%%%%%%%%%%%%%%%%%%%
\label{glossary:communication_context}
\item  ~\hypertarget{glossary:communication_context}{\textbf{communication context}} \\*
A prescribed ``communication universe'' that accommodates messages to be 
received within the \emph{communication context} they were sent, and messages 
sent in different \emph{communication contexts} do not interfere.
The communicator also specifies the set of processes that share this
communication context.
Contexts provide the ability to have
a separate safe ``universe''
of message-passing between the two groups.  
A context is akin to an additional
tag that differentiates messages.
The system manages this differentiation process.
The use of separate communication
contexts by distinct libraries (or distinct library invocations)
insulates communication internal to the library execution from
external communication.  This allows the invocation of the library even if
there are pending communications
on ``other'' communicators, and avoids the need to
synchronize entry or exit into library code.  Pending point-to-point
communications are also guaranteed not to interfere with collective
communications within a single communicator.
 Refer to Section~\ref{subsec:pt2pt-envelope} on page~\pageref{subsec:pt2pt-envelope},
 and to Section~\ref{sec:context} on page~\pageref{sec:context}.

%%%%%%%%%%%%%%%%%%%%%%%%%%%%%%%%%%%%%%%%%%%%%%%%%%%%%%%%%%%%%%%%%%%%%%
\label{glossary:commutative}
\item  ~\hypertarget{glossary:commutative}{\textbf{commutative}} \\*
The property of a collective reduction that changing the order of the operands does not change the end result. 
Binary operation $ \odot $ is commutative iff:
\begin{eqnarray}
A \odot B = B \odot A
\end{eqnarray}
Reduction operation properties apply to \func{MPI\_OP\_CREATE} argument MPI\_Op~*op.
(See also ~\hyperlink{glossary:associative}{associative}.)
Refer to Section~\ref{subsec:coll-user-ops} on page~\pageref{subsec:coll-user-ops}.

%%%%%%%%%%%%%%%%%%%%%%%%%%%%%%%%%%%%%%%%%%%%%%%%%%%%%%%%%%%%%%%%%%%%%%
\label{glossary:completion_completed}
\item  ~\hypertarget{glossary:completion_completed}{\textbf{completion/completed}} \\*
The word \emph{complete} is used with respect to operations, requests, and communications. 
An operation completes when the user is allowed to reuse resources, and any output 
buffers have been updated; i.e. a call to \func{MPI\_TEST} will return flag = true. A request is 
completed by a call to \func{MPI\_WAIT}, which returns, or a \func{MPI\_TEST} or \func{MPI\_GET\_STATUS} 
call which returns flag = true. 
This completing call has two effects: the status is extracted from the request; in the case 
of test and wait, if the request was non persistent, it is freed, and becomes inactive if it 
was persistent. A communication completes when all participating operations complete.
Refer to Section~\ref{terms:semantic} on page~\pageref{terms:semantic}.

%%%%%%%%%%%%%%%%%%%%%%%%%%%%%%%%%%%%%%%%%%%%%%%%%%%%%%%%%%%%%%%%%%%%%%
\label{glossary:contiguous}
\item  ~\hypertarget{glossary:contiguous}{\textbf{contiguous}} \\*
A collection of memory locations that are adjacent to one another
without intervening data.
Refer to Section~\ref{subsec:pt2pt-datatypeconst} on page~\pageref{subsec:pt2pt-datatypeconst}.

%%%%%%%%%%%%%%%%%%%%%%%%%%%%%%%%%%%%%%%%%%%%%%%%%%%%%%%%%%%%%%%%%%%%%%
\label{glossary:conversion}
\item  ~\hypertarget{glossary:conversion}{\textbf{conversion}} \\*
See ~\hypertarget{glossary:type_conversion}{\textbf{type conversion}} and ~\hypertarget{glossary:representation_conversion}{\textbf{type conversion}}.

%%%%%%%%%%%%%%%%%%%%%%%%%%%%%%%%%%%%%%%%%%%%%%%%%%%%%%%%%%%%%%%%%%%%%%
\label{glossary:correct_program}
\item  ~\hypertarget{glossary:correct_program}{\textbf{correct program}} \\*
A program that performs as intended by the program developer; A program that is free of bugs. 
For example, a  {\bf correct program} that utilizes collective communications
must invoke collective communications so
that deadlock will
not occur, whether collective communications are synchronizing or not.
(See also ~\hyperlink{glossary:erroneous_program}{erroneous program}.)
Refer to Section~\ref{coll:correct} on page~\pageref{coll:correct}.

%%%%%%%%%%%%%%%%%%%%%%%%%%%%%%%%%%%%%%%%%%%%%%%%%%%%%%%%%%%%%%%%%%%%%%
\label{glossary:created_datatype}
\item  ~\hypertarget{glossary:created_datatype}{\textbf{created datatype}} \\*
The initial step in preparing a \emph{derived datatype} (before ``committed datatype'').
(See also ~\hyperlink{glossary:committed_datatype}{committed datatype}.)
Refer to Section~\ref{subsec:pt2pt-comfree} on page~\pageref{subsec:pt2pt-comfree}.

%%%%%%%%%%%%%%%%%%%%%%%%%%%%%%%%%%%%%%%%%%%%%%%%%%%%%%%%%%%%%%%%%%%%%%
\label{glossary:data_conversion}
\item  ~\hypertarget{glossary:data_conversion}{\textbf{data conversion}} \\*
See ~\hypertarget{glossary:type_conversion}{\textbf{type conversion}} and ~\hypertarget{glossary:representation_conversion}{\textbf{type conversion}}.

%%%%%%%%%%%%%%%%%%%%%%%%%%%%%%%%%%%%%%%%%%%%%%%%%%%%%%%%%%%%%%%%%%%%%%
\label{glossary:datatype}
\item  ~\hypertarget{glossary:datatype}{\textbf{datatype}} \\*
A \emph{datatype} defines a mechanism to describe any data layout, e.g., an array of
structures in the memory, which can be used as a message send or receive buffer.
(See also ~\hyperlink{glossary:basic_datatype}{basic datatype} and
~\hyperlink{glossary:derived_datatype}{derived datatype}.)
Refer to Chapter ~\ref{chap:datatypes} on page~\pageref{chap:datatypes}.

%%%%%%%%%%%%%%%%%%%%%%%%%%%%%%%%%%%%%%%%%%%%%%%%%%%%%%%%%%%%%%%%%%%%%%
\label{glossary:datatype_handle}
\item  ~\hypertarget{glossary:datatype_handle}{\textbf{datatype handle}} \\*
A datatype handle is use to describe the buffer in a communication call.
Refer to Section~\ref{subsec:pt2pt-comfree} on page~\pageref{subsec:pt2pt-comfree}.

%%%%%%%%%%%%%%%%%%%%%%%%%%%%%%%%%%%%%%%%%%%%%%%%%%%%%%%%%%%%%%%%%%%%%%
\label{glossary:datatype_optional}
\item  ~\hypertarget{glossary:datatype_optional}{\textbf{datatype, optional}} \\*
A type that is not required by conforming implementations of \MPIIII/ (e.g., \type{MPI\_INTEGER2}).
Refer to Section~\ref{subsec:pt2pt-comfree} on page~\pageref{subsec:pt2pt-comfree}.

%%%%%%%%%%%%%%%%%%%%%%%%%%%%%%%%%%%%%%%%%%%%%%%%%%%%%%%%%%%%%%%%%%%%%%
\label{glossary:deprecated}
\item  ~\hypertarget{glossary:deprecated}{\textbf{deprecated}} \\*
Constructs that continue to be part of the \MPI/ standard, 
as documented in Chapter~\ref{chap:deprecated}, 
but that users are recommended not to continue using, since 
better solutions were provided with the current standard.
Note that deprecated constructs may eventually be withdrawn.
For example, the Fortran binding 
for \mpii/ functions that have address arguments uses {\tt INTEGER}.
This is not consistent with the C binding, and causes problems on
machines with 32 bit {\tt INTEGER}s and 64 bit addresses.
Refer to Section~\ref{sec:deprecated} on page~\pageref{sec:deprecated}.

%%%%%%%%%%%%%%%%%%%%%%%%%%%%%%%%%%%%%%%%%%%%%%%%%%%%%%%%%%%%%%%%%%%%%%
\label{glossary:derived_datatype}
\item  ~\hypertarget{glossary:derived_datatype}{\textbf{derived datatype}} \\*
A derived datatype is any datatype that is not predefined.
They are opaque objects that specifies two
things:
\begin{itemize}
\item
A sequence of basic datatypes
\item
A sequence of integer (byte) displacements
\end{itemize}
The displacements are not required to be positive, distinct, or
in increasing order. Therefore, the order of items need not
coincide with their order in store, and an item may appear more than
once. (See also ~\hyperlink{glossary:general_datatype}{general datatype}, ~\hyperlink{glossary:basic_datatype}{basic datatype} 
and  ~\hyperlink{glossary:typemap}{typemap}.)
Refer to Section~\ref{terms:semantic} on page~\pageref{terms:semantic} and 
Chapter ~\ref{chap:datatypes} on page~\pageref{chap:datatypes}.

%%%%%%%%%%%%%%%%%%%%%%%%%%%%%%%%%%%%%%%%%%%%%%%%%%%%%%%%%%%%%%%%%%%%%%
\label{glossary:displacement_file}
\item  ~\hypertarget{glossary:displacement_file}{\textbf{displacement, file}} \\*
A file {\it displacement} is an absolute byte position
relative to the beginning of a file.
The displacement defines the location where a {\it view} begins.
% A displacement is an offset relative to the default view (see below).
Note that a ``file displacement'' is distinct from a ``typemap displacement.''
Refer to Section~\ref{subsec:io-2:definitions} on page~\pageref{subsec:io-2:definitions}.

%%%%%%%%%%%%%%%%%%%%%%%%%%%%%%%%%%%%%%%%%%%%%%%%%%%%%%%%%%%%%%%%%%%%%%
\label{glossary:end_of_file}
\item  ~\hypertarget{glossary:end_of_file}{\textbf{end of file}} \\*
See ~\hyperlink{glossary:file_size}{file size}.

%%%%%%%%%%%%%%%%%%%%%%%%%%%%%%%%%%%%%%%%%%%%%%%%%%%%%%%%%%%%%%%%%%%%%%
\label{glossary:equivalent_datatype}
\item  ~\hypertarget{glossary:equivalent_datatype}{\textbf{equivalent datatype}} \\*
Two datatypes are equivalent if they appear to have been created with
the same sequence of calls (and arguments) and thus have the same
typemap.  Two equivalent datatypes do not necessarily have the same
cached attributes or the same names.
Refer to Section~\ref{terms:semantic} on page~\pageref{terms:semantic}.

%%%%%%%%%%%%%%%%%%%%%%%%%%%%%%%%%%%%%%%%%%%%%%%%%%%%%%%%%%%%%%%%%%%%%%
\label{glossary:erroneous_program}
\item  ~\hypertarget{glossary:erroneous_program}{\textbf{erroneous program}} \\*
A program that does not perform as intended by the program developer; A program that contains bugs. 
(See also ~\hyperlink{glossary:correct_program}{correct program}.)
Refer to Section~\ref{coll:correct} on page~\pageref{coll:correct} and
Section 2.8.

%%%%%%%%%%%%%%%%%%%%%%%%%%%%%%%%%%%%%%%%%%%%%%%%%%%%%%%%%%%%%%%%%%%%%%
\label{glossary:error_class}
\item  ~\hypertarget{glossary:error_class}{\textbf{error class}} \\*
A construct designed to allow implementations freedom with their choice
of error codes while at the same time provide a predefined subset of error
codes to permit an application to classify a specific error code.
This is done to allow an implementation to
provide as much information as possible in the error code (for use with
\mpifunc{MPI\_ERROR\_STRING}).
To make it possible for an application to interpret an error code, the routine
\mpifunc{MPI\_ERROR\_CLASS} %mansplit
converts any error code into one of a small set of standard error
codes, called {\em error classes}.  
% Valid error classes include
Valid error classes are shown in Table~\ref{table:inquiry:errclasses:part:i}
and Table~\ref{table:inquiry:errclasses:part:ii}. 
Refer to Section~\ref{sec:errorhandler} on page~\pageref{sec:errorhandler}.

%%%%%%%%%%%%%%%%%%%%%%%%%%%%%%%%%%%%%%%%%%%%%%%%%%%%%%%%%%%%%%%%%%%%%%
\label{glossary:etype}
\item  ~\hypertarget{glossary:etype}{\textbf{etype}} \\*
An {\it etype} ({\it elementary} datatype)
is the unit of data access and positioning for file I/O.
It can be any \MPI/ predefined or derived datatype.
Derived etypes can be constructed
using any of the \MPI/ datatype constructor routines,
provided all resulting typemap displacements are non-negative
and monotonically nondecreasing.
Data access is performed in etype units,
reading or writing whole data items of type etype.
Offsets are expressed as a count of etypes;
file pointers point to the beginning of etypes.
Depending on context,
the term ``etype'' is used to describe one of three aspects
of an elementary datatype:
a particular \MPI/ type,
a data item of that type,
or the extent of that type.
Refer to Section~\ref{subsec:io-2:definitions} on page~\pageref{subsec:io-2:definitions}.

%%%%%%%%%%%%%%%%%%%%%%%%%%%%%%%%%%%%%%%%%%%%%%%%%%%%%%%%%%%%%%%%%%%%%%
\label{glossary:exscan_operation}
\item  ~\hypertarget{glossary:exscan_operation}{\textbf{exscan operation}} \\*
A collective procedure in which a partial reduction operation is performed on data supplied by the members of a group.
In the exclusive scan (see \func{MPI\_EXSCAN}), the prefix reduction on process i only includes data up to i-1. 
(See also ~\hyperlink{glossary:scan_operation}{scan operation} and
~\hyperlink{glossary:reduce_operation}{reduce operation}.)
Refer to Section~\ref{global-reduce} on page~\pageref{global-reduce}.

%%%%%%%%%%%%%%%%%%%%%%%%%%%%%%%%%%%%%%%%%%%%%%%%%%%%%%%%%%%%%%%%%%%%%%
\label {glossary:exposure_epoch}
\item  ~\hypertarget{glossary:exposure_epoch}{\textbf{exposure epoch}} \\*
The period during which a target window can be accessed by \RMA/
operations in {\it active target} communication.
(See also ~\hyperlink{glossary:access_epoch}{access epoch},
~\hyperlink{glossary:active_target}{active target}
 and ~\hyperlink{glossary:RMA}{RMA}.)
Refer to Section~\ref{sec:1sided-sync} on page~\pageref{sec:1sided-sync}.

%%%%%%%%%%%%%%%%%%%%%%%%%%%%%%%%%%%%%%%%%%%%%%%%%%%%%%%%%%%%%%%%%%%%%%
\label{glossary:extent}
\item  ~\hypertarget{glossary:extent}{\textbf{extent}} \\*
The {\bf extent} of a datatype is defined to
be the span from the first byte to the last byte occupied by entries in this
datatype, rounded up to satisfy alignment requirements.
That is, if
\[
Typemap = \{ (type_0,disp_0), ..., (type_{n-1}, disp_{n-1}) \} ,
\]
then
\begin{eqnarray}
lb(Typemap) & = & \min_j disp_j , \nonumber \\
ub(Typemap) & = & \max_j (disp_j + sizeof(type_j)) + \epsilon , \mbox{ and}
\nonumber \\ extent(Typemap) & = & ub(Typemap) -lb(Typemap).
\end{eqnarray}
If $type_i$ requires alignment to a byte address that 
% is is 
is
a multiple
of $k_i$,
then $\epsilon$ is the least non-negative increment needed to round
$extent(Typemap)$ to the next multiple of $\max_i k_i$.
For datatypes that 
have a ``hole'' at its beginning or its end, or a datatype with
entries that extend above the upper bound or below the lower bound, then
\label{eq:pt2pt-extent}
\[
extent(Typemap) = ub(Typemap) - lb(Typemap)
\]
(See also ~\hyperlink{glossary:lower_bound}{lower bound}, ~\hyperlink{glossary:upper_bound}{upper bound}, and ~\hyperlink{glossary:true_extent}{true extent}.)
Refer to Section~\ref{sec:pt2pt-datatype} on page~\pageref{sec:pt2pt-datatype}.

%%%%%%%%%%%%%%%%%%%%%%%%%%%%%%%%%%%%%%%%%%%%%%%%%%%%%%%%%%%%%%%%%%%%%%
\label{glossary:external32}
\item  ~\hypertarget{glossary:external32}{\textbf{external32}} \\*
External32 is a data representation format/specification.
The data on the storage
medium is always in this canonical External32 representation, and
the data in memory
is always in the local process's native representation.
% is always in the representation of the process that placed it there.
(See also ~\hyperlink{glossary:native}{native} and ~\hyperlink{glossary:internal}{internal}.)
Refer to Section~\ref{sec:io-file-interop} on page~\pageref{sec:io-file-interop}.

%%%%%%%%%%%%%%%%%%%%%%%%%%%%%%%%%%%%%%%%%%%%%%%%%%%%%%%%%%%%%%%%%%%%%%
\label{glossary:fairness}
\item  ~\hypertarget{glossary:fairness}{\textbf{fairness}} \\*
The property of parallel and distributed systems that no process is starved, 
and all processes are accorded the same priority in allowing their accesses 
to shared resources. When fairness is imposed, all processes have the 
chance to make progress regardless of what other processes may be 
doing at the same time. Note that \MPI/  makes no fairness guarantees.
Suppose that a send is posted.  Then it is possible
that the destination process repeatedly posts a receive that matches this
send (e.g., using tag \const{MPI\_ANY\_SOURCE}), yet the message is never received, because in each case it is overtaken by
another message, sent from another source.  Similarly, suppose that a
receive was posted by a multi-threaded process.  Then it is possible that
messages that
match this receive are repeatedly received, yet the receive is never satisfied,
because it is overtaken by other receives posted at this process (by
other executing threads).  It is the programmer's responsibility to prevent
starvation in such situations.
Refer to Section~\ref{sec:pt2pt-semantics} on page~\pageref{sec:pt2pt-semantics}.

%%%%%%%%%%%%%%%%%%%%%%%%%%%%%%%%%%%%%%%%%%%%%%%%%%%%%%%%%%%%%%%%%%%%%%
\label{glossary:file}
\item  ~\hypertarget{glossary:file}{\textbf{file}} \\*
An \MPI/ file is an ordered collection of typed data items.
\MPI/ supports random or sequential access to any integral set of these items.
% An \MPI/ file is an ordered collection of data bytes
% supporting random or sequential access to any integral set of bytes.
A file is opened collectively by a group of processes.
All collective I/O calls on a file are collective over this group.
Refer to Section~\ref{subsec:io-2:definitions} on page~\pageref{subsec:io-2:definitions}.

%%%%%%%%%%%%%%%%%%%%%%%%%%%%%%%%%%%%%%%%%%%%%%%%%%%%%%%%%%%%%%%%%%%%%%
\label{glossary:file_consistency}
\item  ~\hypertarget{glossary:file_consistency}{\textbf{file consistency}} \\*
Consistency semantics define the outcome of multiple accesses
to a single file.
All file accesses in \MPI/ are relative to a specific file handle
created from a collective open.
\MPI/ provides three levels of consistency:
sequential consistency among all accesses using a single file handle,
sequential consistency among all accesses
using file handles created from a single collective open
with atomic mode enabled,
and
user-imposed consistency among accesses other than the above.
%--via the \func{MPI\_FILE\_SYNC} routine.
%-weak consistency among accesses not handled above
%-via the \func{MPI\_FILE\_SYNC} routine.
Sequential consistency means the behavior of a set of operations
will be as if the operations were performed in some serial order
consistent with program order; each access appears atomic,
although the exact ordering of accesses is unspecified.
User-imposed consistency may be obtained using program order
and calls to \func{MPI\_FILE\_SYNC}.
Refer to Section~\ref{sec:io-semantics} on page~\pageref{sec:io-semantics}.

%%%%%%%%%%%%%%%%%%%%%%%%%%%%%%%%%%%%%%%%%%%%%%%%%%%%%%%%%%%%%%%%%%%%%%
\label{glossary:file_handle}
\item  ~\hypertarget{glossary:file_handle}{\textbf{file handle}} \\*
A reference to an opaque structure that enables a program to access a particular file.
A {\it file handle} is an opaque object created by \func{MPI\_FILE\_OPEN}
and freed by \func{MPI\_FILE\_CLOSE}.
All operations on an open file
reference the file through the file handle.
Refer to Section~\ref{subsec:io-2:definitions} on page~\pageref{subsec:io-2:definitions}.

%%%%%%%%%%%%%%%%%%%%%%%%%%%%%%%%%%%%%%%%%%%%%%%%%%%%%%%%%%%%%%%%%%%%%%
\label{glossary:file_interoperability}
\item  ~\hypertarget{glossary:file_interoperability}{\textbf{file interoperability}} \\*
file interoperability is the ability to
correctly interpret the representation of
 information previously written to a file.
\MPI/ supports the External32 data representation (Section~\ref{subsec:ext32}, page~\pageref{subsec:ext32}) as
well as the data conversion functions (Section~\ref{sec:io-datarep},
page~\pageref{sec:io-datarep}). 
Refer to Section~\ref{sec:io-file-interop} on page~\pageref{sec:io-file-interop}.

%%%%%%%%%%%%%%%%%%%%%%%%%%%%%%%%%%%%%%%%%%%%%%%%%%%%%%%%%%%%%%%%%%%%%%
\label{glossary:file_pointer}
\item  ~\hypertarget{glossary:file_pointer}{\textbf{file pointer}} \\*
A {\it file pointer} is an implicit offset maintained by \MPI/.
``Individual file pointers'' are file pointers that are local to
each process that opened the file.
A ``shared file pointer'' is a file pointer that is shared by
the group of processes that opened the file.
Refer to Section~\ref{subsec:io-2:definitions} on page~\pageref{subsec:io-2:definitions}.

%%%%%%%%%%%%%%%%%%%%%%%%%%%%%%%%%%%%%%%%%%%%%%%%%%%%%%%%%%%%%%%%%%%%%%
\label{glossary:file_size}
\item  ~\hypertarget{glossary:file_size}{\textbf{file size}} \\*
The {\it size} of an \MPI/ file is measured in bytes from the 
beginning of the file.  A newly created file has a size of zero 
bytes.  
If \func{MPI\_FILE\_SET\_SIZE} has resized the file past the
current file size, the values of data in the new regions in the file (those
locations with displacements between old file size and the specified new file \textbf{\mpiarg{size}})
are undefined.
It is implementation dependent whether the \func{MPI\_FILE\_SET\_SIZE} routine
allocates file space---use \func{MPI\_FILE\_PREALLOCATE}
to force file space to be reserved.
Refer to Section~\ref{subsec:io-2:definitions} on page~\pageref{subsec:io-2:definitions}.

%%%%%%%%%%%%%%%%%%%%%%%%%%%%%%%%%%%%%%%%%%%%%%%%%%%%%%%%%%%%%%%%%%%%%%
\label{glossary:file_view}
\item  ~\hypertarget{glossary:file_view}{\textbf{file view}} \\*
See ~\hyperlink{glossary:view_file}{view, file} .

%%%%%%%%%%%%%%%%%%%%%%%%%%%%%%%%%%%%%%%%%%%%%%%%%%%%%%%%%%%%%%%%%%%%%%
\label{glossary:filetype}
\item  ~\hypertarget{glossary:filetype}{\textbf{filetype}} \\*
A {\it filetype} is the basis for partitioning a file among processes
and defines a template for accessing the file.
A filetype is either a single etype or a derived \MPI/ datatype
constructed from multiple instances of the same etype.
In addition,
the extent of any hole in the filetype
must be a multiple of the etype's extent.
The displacements in the typemap of the filetype are not required to be distinct,
Refer to Section~\ref{subsec:io-2:definitions} on page~\pageref{subsec:io-2:definitions}.

%%%%%%%%%%%%%%%%%%%%%%%%%%%%%%%%%%%%%%%%%%%%%%%%%%%%%%%%%%%%%%%%%%%%%%
\label{glossary:gather}
\item  ~\hypertarget{glossary:gather}{\textbf{gather}} \\*
A collective operation in which one message is sent from \emph{\textbf{n}} 
processes in the group to the root. Messages are concatenated in rank order, and the
resulting message is received by the root as if by a call to
\mpifunc{MPI\_RECV(recvbuf, recvcount$\cdot$n, recvtype, ...)}.
Refer to Section~\ref{sec:coll-intro} on page~\pageref{sec:coll-intro} and  
Section~\ref{sec:coll-gather} on page~\pageref{sec:coll-gather}.

%%%%%%%%%%%%%%%%%%%%%%%%%%%%%%%%%%%%%%%%%%%%%%%%%%%%%%%%%%%%%%%%%%%%%%
\label{glossary:general_datatype}
\item  ~\hypertarget{glossary:general_datatype}{\textbf{general datatype}} \\*
see ~\hyperlink{glossary:derived_datatype}{derived datatype}.

%%%%%%%%%%%%%%%%%%%%%%%%%%%%%%%%%%%%%%%%%%%%%%%%%%%%%%%%%%%%%%%%%%%%%%
\label{glossary:generalized_request}
\item  ~\hypertarget{glossary:generalized_request}{\textbf{generalized request}} \\*
The request object associated with a user-defined non-blocking operation.
Refer to Section~\ref{sec:ei-intro} on page~\pageref{sec:ei-intro} 
~\ref{sec:sec:ei-gr} on page~\pageref{sec:ei-gr}.

%%%%%%%%%%%%%%%%%%%%%%%%%%%%%%%%%%%%%%%%%%%%%%%%%%%%%%%%%%%%%%%%%%%%%%
\label{glossary:global}
\item  ~\hypertarget{glossary:global}{\textbf{global}} \\*
Referring to all members of a group.
Refer to Section~\ref{sec:coll-intro} on page~\pageref{sec:coll-intro}.

%%%%%%%%%%%%%%%%%%%%%%%%%%%%%%%%%%%%%%%%%%%%%%%%%%%%%%%%%%%%%%%%%%%%%%
\label{glossary:graph_topology}
\item  ~\hypertarget{glossary:graph_topology}{\textbf{graph topology}} \\*
A virtual ordering of the processes according to a generalized graph.
(See also ~\hyperlink{glossary:topology}{topology} and ~\hyperlink{glossary:cartesian_topology}{cartesian topology}.)
Refer to Section~\ref{chap:topol} on page~\pageref{chap:topol}.

%%%%%%%%%%%%%%%%%%%%%%%%%%%%%%%%%%%%%%%%%%%%%%%%%%%%%%%%%%%%%%%%%%%%%%
\label{glossary:groups}
\item  ~\hypertarget{glossary:groups}{\textbf{groups}} \\*
Groups define an ordered collection of processes, each with a rank, and it is this
group that defines the low-level \emph{identifiers} for inter-process communication (ranks
are used for sending and receiving).  Thus, groups define a scope for process
names in point-to-point communication.  
%% In addition, groups define the scope of collective operations.  (from Torsten Hoefler: not sure I agree. )
Groups may be manipulated separately from
communicators in \MPI/. While it is not permissible to communicate over just a group,
a group may be used in one-sided communications to restrict the members associated with an epoch. 
Each process in
the group is assigned a rank between {\tt 0} and {\tt n-1}.
(See also ~\hyperlink{glossary:topology}{topology} .)
Refer to Section ~\ref{subsec:pt2pt-envelope} on page~\pageref{subsec:pt2pt-envelope}
and Chapter~\ref{chap:context} on page~\pageref{chap:context}.

%%%%%%%%%%%%%%%%%%%%%%%%%%%%%%%%%%%%%%%%%%%%%%%%%%%%%%%%%%%%%%%%%%%%%%
\label{glossary:handle}
\item  ~\hypertarget{glossary:handle}{\textbf{handle}} \\*
A value that enables a program to access a particular opaque structure.
Opaque structures are used in \MPI/ for communicators, files, requests, etc.
Refer to Section~\ref{terms:opaque-objects} on page~\pageref{terms:opaque-objects}.

%%%%%%%%%%%%%%%%%%%%%%%%%%%%%%%%%%%%%%%%%%%%%%%%%%%%%%%%%%%%%%%%%%%%%%
\label{glossary:implementation}
\item  ~\hypertarget{glossary:implementation}{\textbf{implementation}} \\*
A specific fulfillment of a specification.  
Refer to Section~\ref{terms:semantic} on page~\pageref{terms:semantic}.

%%%%%%%%%%%%%%%%%%%%%%%%%%%%%%%%%%%%%%%%%%%%%%%%%%%%%%%%%%%%%%%%%%%%%%
\label{glossary:IN}
\item  ~\hypertarget{glossary:IN}{\textbf{IN}} \\*
An argument of an \MPI/ procedure call with the following property: the call may use the input value but does 
not update the argument. 
(See also ~\hyperlink{glossary:INOUT}{INOUT} and ~\hyperlink{glossary:OUT}{OUT}.)
%%% Note: this should be section 2.3, but there is no appropriate label
Refer to Section~\ref{terms:semantic} on page~\pageref{terms:semantic}.

%%%%%%%%%%%%%%%%%%%%%%%%%%%%%%%%%%%%%%%%%%%%%%%%%%%%%%%%%%%%%%%%%%%%%%
\label{glossary:INOUT}
\item  ~\hypertarget{glossary:INOUT}{\textbf{INOUT}} \\*
An argument of an \MPI/ procedure call with the following property: the call may both use and update the argument.
(See also ~\hyperlink{glossary:IN}{IN} and ~\hyperlink{glossary:OUT}{OUT}.)
%%% Note: this should be section 2.3, but there is no appropriate label
Refer to Section~\ref{terms:semantic} on page~\pageref{terms:semantic}.

%%%%%%%%%%%%%%%%%%%%%%%%%%%%%%%%%%%%%%%%%%%%%%%%%%%%%%%%%%%%%%%%%%%%%%
\label{glossary:in_place}
\item  ~\hypertarget{glossary:in_place}{\textbf{in place}} \\*
Communication in which the input buffer is reused as the output buffer.
This is specified by
providing a special argument value, \const{MPI\_IN\_PLACE}, instead of the
send buffer or the receive buffer argument,
depending on the operation performed. 
Refer to Section~\ref{sec:coll-communicator} on page~\pageref{sec:coll-communicatorr}.

%%%%%%%%%%%%%%%%%%%%%%%%%%%%%%%%%%%%%%%%%%%%%%%%%%%%%%%%%%%%%%%%%%%%%%
\label{glossary:intercommunicator}
\item  ~\hypertarget{glossary:intercommunicator}{\textbf{intercommunicator}} \\*
A communicator that identifies two distinct groups (local and remote) of processes
linked with a context.  (See also ~\hyperlink{glossary:intracommunicator}{intracommunicator}.)
Refer to Section~\ref{sec:coll-communicator} on page~\pageref{sec:coll-communicator}  
and Section~\ref{sec:context-intercom} on page~\pageref{sec:context-intercom}.

%%%%%%%%%%%%%%%%%%%%%%%%%%%%%%%%%%%%%%%%%%%%%%%%%%%%%%%%%%%%%%%%%%%%%%
\label{glossary:interface}
\item  ~\hypertarget{glossary:interface}{\textbf{interface}} \\*
Syntax and semantics for invoking services from within an executing application.  
Refer to Section~\ref{terms:semantic} on page~\pageref{terms:semantic}.

%%%%%%%%%%%%%%%%%%%%%%%%%%%%%%%%%%%%%%%%%%%%%%%%%%%%%%%%%%%%%%%%%%%%%%
\label{glossary:internal_data_representation}
\item  ~\hypertarget{glossary:internal_data_representation}{\textbf{internal data representation}} \\*
Data in \emph{internal} data representation can be used for I/O operations in a homogeneous or
heterogeneous environment; the implementation will perform type
conversions if necessary. The implementation is free to store data in
any format of its choice,
with the restriction that it will maintain constant extents
for all predefined datatypes in any one file.
The environment in which the resulting file 
can be reused is implementation-defined
and must be documented by the implementation.
(See also ~\hyperlink{glossary:native_data_representation}{native data representation} and ~\hyperlink{glossary:external32}{external32}.)
Refer to Section~\ref{sec:io-file-interop} on page~\pageref{sec:io-file-interop}.

%%%%%%%%%%%%%%%%%%%%%%%%%%%%%%%%%%%%%%%%%%%%%%%%%%%%%%%%%%%%%%%%%%%%%%
\label{glossary:intracommunicator}
\item  ~\hypertarget{glossary:intracommunicator}{\textbf{intracommunicator}} \\*
A communicator in which the communication domain does not extend beyond the members of a single local group. 
(See also ~\hyperlink{glossary:intercommunicator}{intercommunicator}.)
Refer to Section~\ref{sec:coll-communicator} on page~\pageref{sec:coll-communicator}.

%%%%%%%%%%%%%%%%%%%%%%%%%%%%%%%%%%%%%%%%%%%%%%%%%%%%%%%%%%%%%%%%%%%%%%
\label{glossary:local_group}
\item  ~\hypertarget{glossary:local_group}{\textbf{local group}} \\*
The recipients of a procedure for an intercommunicator. 
(See also ~\hyperlink{glossary:intercommunicator}{intercommunicator} and
~\hyperlink{glossary:remote_group}{remote group}.)
Refer to Section~\ref{sec:coll-communicator} on page~\pageref{sec:coll-communicator}.

%%%%%%%%%%%%%%%%%%%%%%%%%%%%%%%%%%%%%%%%%%%%%%%%%%%%%%%%%%%%%%%%%%%%%%
\label{glossary:local operation}
\item  ~\hypertarget{glossary:local_operation}{\textbf{local operation}} \\*
A procedure is local if 
it returns irrespective of the status of other processes.
(See also ~\hyperlink{glossary:non-local_operation}{non-local operation}.)
Refer to Section~\ref{terms:semantic} on page~\pageref{terms:semantic}.

%%%%%%%%%%%%%%%%%%%%%%%%%%%%%%%%%%%%%%%%%%%%%%%%%%%%%%%%%%%%%%%%%%%%%%
\label{glossary:lower_bound_datatype}
\item  ~\hypertarget{glossary:lower_bound_datatype}{\textbf{lower bound, datatype}} \\*
The displacement of the lowest unit of addressable memory within the datatype. 
%%In general, if
%%\begin{displaymath}
%%Typemap = \{ (type_0 , disp_0 ) , ... , (type_{n-1} , disp_{n-1}) \} ,
%%\end{displaymath}
%%then the {\bf lower bound} of $Typemap$ is defined to be
%%\[
%%lb(Typemap) = \left\{ \begin{array}{ll}
%%\min_j disp_j & \mbox{if no entry has basic type {\sf lb}} \\
%%\min_j \{ disp_j \ \mbox{such that}\ type_j = {\sf lb} \} & \mbox{otherwise}
%%\end{array}
%%\right. \]
(See also ~\hyperlink{glossary:upper_bound}{upper bound} and ~\hyperlink{glossary:extent}{extent}.)
Refer to \emph{lb(Typemap)} in Section~\ref{pt2pt-exX} on page~\pageref{pt2pt-exX}.

%%%%%%%%%%%%%%%%%%%%%%%%%%%%%%%%%%%%%%%%%%%%%%%%%%%%%%%%%%%%%%%%%%%%%%
\label{glossary:matching}
\item  ~\hypertarget{glossary:matching}{\textbf{matching}} \\*
(a) A language type (e.g., float) matches an MPI datatype (e.g., \type{MPI\_FLOAT}). 
(b) Two datatypes match if their type signatures are identical.
(c) A message matches a receive operation, if (c.1) the communicators are identical, 
(c.2) the source ranks either are identical or a wildcard source was specified by the receive operation, 
and (c.3) the tags are identical or a wildcard tag was specified by the receive operation.
 Refer to Section~\ref{subsec:pt2pt-typematch} on page~\pageref{subsec:pt2pt-typematch}.

%%%%%%%%%%%%%%%%%%%%%%%%%%%%%%%%%%%%%%%%%%%%%%%%%%%%%%%%%%%%%%%%%%%%%%
\label{glossary:message_envelope}
\item  ~\hypertarget{glossary:message_envelope}{\textbf{message envelope}} \\*
The non-data portion of a message that can be used to
distinguish messages and selectively receive them.  This information consists
of a fixed number of fields, which are collectively called
the {\bf message envelope}.   These fields are, source, 
destination, tag, communicator, and length.
Refer to Section~\ref{subsec:pt2pt-envelope} on page~\pageref{subsec:pt2pt-envelope}.

%%%%%%%%%%%%%%%%%%%%%%%%%%%%%%%%%%%%%%%%%%%%%%%%%%%%%%%%%%%%%%%%%%%%%%
\label{glossary:native_data_representation}
\item  ~\hypertarget{glossary:native_data_representation}{\textbf{native data representation}} \\*
A format for storing data.
Data in \emph{native} representation is stored in a file exactly
as it is in memory.
The advantage of this data representation is that
data precision and I/O performance are not lost in type conversions
with a purely homogeneous environment.
The disadvantage is the loss of transparent interoperability within a
heterogeneous \MPI/ environment.
(See also ~\hyperlink{glossary:internal_data_representation}{internal data representation} and ~\hyperlink{glossary:external32}{external32}.)
Refer to Section~\ref{sec:io-file-interop} on page~\pageref{sec:io-file-interop}.

%%%%%%%%%%%%%%%%%%%%%%%%%%%%%%%%%%%%%%%%%%%%%%%%%%%%%%%%%%%%%%%%%%%%%%
\label{glossary:non-local_operation}
\item  ~\hypertarget{glossary:non-local_operation}{\textbf{non-local operation}} \\*
The property of a procedure that completion of the procedure may require
the execution of some \MPI/ procedure on another process.  Such a
procedure may require
communication occurring with another user process.
That is, the return of the procedure is dependent on other processes.
(See also ~\hyperlink{glossary:local_operation}{local operation}.)
Refer to Section~\ref{terms:semantic} on page~\pageref{terms:semantic}.

%%%%%%%%%%%%%%%%%%%%%%%%%%%%%%%%%%%%%%%%%%%%%%%%%%%%%%%%%%%%%%%%%%%%%%
\label{glossary:non-overtaking}
\item  ~\hypertarget{glossary:non-overtaking}{\textbf{non-overtaking}} \\*
The requirement that
if a sender sends two messages in succession to the same destination, and
both match the same receive, then the receiver cannot receive the
second message if the first one is still pending.
Refer to Section~\ref{sec:pt2pt-semantics} on page~\pageref{sec:pt2pt-semantics}.

%%%%%%%%%%%%%%%%%%%%%%%%%%%%%%%%%%%%%%%%%%%%%%%%%%%%%%%%%%%%%%%%%%%%%%
\label{glossary:nonblocking}
\item  ~\hypertarget{glossary:nonblocking}{\textbf{nonblocking}} \\*
The property of a procedure that it may return before the
operation completes, and before the user is allowed to reuse
resources (such as buffers) specified in the call.
A nonblocking request is {\bf started} by the call that initiates it, e.g.,
\mpifunc{MPI\_ISEND}.
(See also ~\hyperlink{glossary:completion_completed}{completion/completed})
Refer to Section~\ref{terms:semantic} on page~\pageref{terms:semantic} and
Section~\ref{sec:pt2pt-nonblock} on page~\pageref{sec:pt2pt-nonblock}.

%%%%%%%%%%%%%%%%%%%%%%%%%%%%%%%%%%%%%%%%%%%%%%%%%%%%%%%%%%%%%%%%%%%%%%
\label{glossary:nondeterminism}
\item  ~\hypertarget{glossary:nondeterminism}{\textbf{nondeterminism}} \\*
A nondeterministic program is one in which either (a) repeated executions 
of the program with the same input may yield different results (weak 
nondeterminism); or (b) the sequence of states through which the program 
passes is not uniquely determined by the input even if the results are the 
same (strong nondeterminism). Nondeterminism may originate from the 
use of wildcards, \func{MPI\_CANCEL}, \func{MPI\_WAITANY}, rounding errors in floating-point 
reduction operations, and so on.
Refer to ``message order'' in Section~\ref{sec:pt2pt-semantics} on page~\pageref{sec:pt2pt-semantics}.

%%%%%%%%%%%%%%%%%%%%%%%%%%%%%%%%%%%%%%%%%%%%%%%%%%%%%%%%%%%%%%%%%%%%%%
\label{glossary:null_handle}
\item  ~\hypertarget{glossary:null_handle}{\textbf{null handle}} \\*
A handle with a reserved value indicating that it refers to no object.
(See also ~\hyperlink{glossary:null_process}{null process})
Refer to Section~\ref{subsec:pt2pt-commend} on page~\pageref{subsec:pt2pt-commend}.

%%%%%%%%%%%%%%%%%%%%%%%%%%%%%%%%%%%%%%%%%%%%%%%%%%%%%%%%%%%%%%%%%%%%%%
\label{glossary:null_process}
\item  ~\hypertarget{glossary:null_process}{\textbf{null process}} \\*
A ``dummy'' source or
destination (\const{MPI\_PROC\_NULL}) used
for communication.  This simplifies the code that is needed for dealing with
boundaries, for example, in the case of a non-circular shift done with calls to
send-receive.
The special value \const{MPI\_PROC\_NULL} can be used
instead of a rank wherever a
source or a destination argument is required in a call.   A communication
with process \const{MPI\_PROC\_NULL} has no effect.
(See also ~\hyperlink{glossary:null_handle}{null handle})
Refer to Section~\ref{sec:pt2pt-nullproc} on page~\pageref{sec:pt2pt-nullproc}.

%%%%%%%%%%%%%%%%%%%%%%%%%%%%%%%%%%%%%%%%%%%%%%%%%%%%%%%%%%%%%%%%%%%%%%
\label{glossary:null_request}
\item  ~\hypertarget{glossary:null_request}{\textbf{null request}} \\*
A request handle with a reserved value (\const{MPI\_REQUEST\_NULL}) indicating that it refers to no object.
(See also ~\hyperlink{glossary:null_handle}{null handle})
Refer to Section~\ref{subsec:pt2pt-commend} on page~\pageref{subsec:pt2pt-commend}.

%%%%%%%%%%%%%%%%%%%%%%%%%%%%%%%%%%%%%%%%%%%%%%%%%%%%%%%%%%%%%%%%%%%%%%
\label{glossary:offset}
\item  ~\hypertarget{glossary:offset}{\textbf{offset}} \\*
An {\it offset} is a position
in the file
relative to the current view,
expressed as a count of etypes.
Holes in the view's filetype are skipped when calculating this position.
Offset 0 is the location of the first etype visible in the view
(after skipping the displacement and any initial holes in the view).
% The beginning of the view,
% which is {\it displacement} bytes from the beginning of the file,
% is at offset 0.
% Holes in the view's filetype are ignored when calculating this position.
For example, an offset of 2 for process 1
in Figure~\ref{fig:gloss-io-comp-filetypes} is the position
of the 8th etype in the file after the displacement.
An ``explicit offset'' is an offset that is used as a formal parameter
in explicit data access routines.
Refer to Section~\ref{subsec:io-2:definitions} on page~\pageref{subsec:io-2:definitions}.

%%%%%%%%%%%%%%%%%%%%%%%%%%%%%%%%%%%%%%%%%%%%%%%%%%%%%%%%%%%%%%%%%%%%%%
\label{glossary:one-sided_communication}
\item  ~\hypertarget{glossary:one-sided_communication}{\textbf{one-sided communication}} \\*
See ~\hyperlink{glossary:RMA}{RMA}.
Refer to Chapter ~\ref{chap:one-side-2} on page~\pageref{chap:one-side-2}.


%%%%%%%%%%%%%%%%%%%%%%%%%%%%%%%%%%%%%%%%%%%%%%%%%%%%%%%%%%%%%%%%%%%%%%
\label{glossary:opaque}
\item  ~\hypertarget{glossary:opaque}{\textbf{opaque}} \\*
Data structures managed by MPI whose size and shape are not visible to the user. 
Opaque structures are accessed by the application program through {\bf handles}, e.g., a communicator handle. 
Refer to Section~\ref{terms:opaque-objects} on page~\pageref{terms:opaque-objects}.

%%%%%%%%%%%%%%%%%%%%%%%%%%%%%%%%%%%%%%%%%%%%%%%%%%%%%%%%%%%%%%%%%%%%%%
\label{glossary:order}
\item  ~\hypertarget{glossary:order}{\textbf{order}} \\*
The arrangement of operations relative to each other according to sequence of invocation.
\MPI/ places several requirements upon order: (a) Point-to-point messages are ~\hyperlink{glossary:non-overtaking}{non-overtaking};
(b) Nonblocking communication operations are ordered to the execution order;
(c) Collective operations must be exectued in the same order at all members of the communication group.
Refer to Section~\ref{subsec:pt2pt-semantics} on page~\pageref{subsec:pt2pt-semantics}.

%%%%%%%%%%%%%%%%%%%%%%%%%%%%%%%%%%%%%%%%%%%%%%%%%%%%%%%%%%%%%%%%%%%%%%
\label{glossary:origin_process}
\item  ~\hypertarget{glossary:origin_process}{\textbf{origin process}} \\*
The process that performs the call in one-sided communications.
(See also ~\hyperlink{glossary:RMA}{RMA} and ~\hyperlink{glossary:target_process}{target process})
Refer to Section~\ref{sec:one-side-2} on page~\pageref{sec:one-side-2}.

%%%%%%%%%%%%%%%%%%%%%%%%%%%%%%%%%%%%%%%%%%%%%%%%%%%%%%%%%%%%%%%%%%%%%%
\label{glossary:OUT}
\item  ~\hypertarget{glossary:OUT}{\textbf{OUT}} \\*
An argument of an \MPI/ procedure call with the following property: the call may update the argument but does not use its input value.
(See also ~\hyperlink{glossary:IN}{IN} and ~\hyperlink{glossary:INOUT}{INOUT}.)
Refer to Section~\ref{terms:semantic} on page~\pageref{terms:semantic}.

%%%%%%%%%%%%%%%%%%%%%%%%%%%%%%%%%%%%%%%%%%%%%%%%%%%%%%%%%%%%%%%%%%%%%%
\label{glossary:pack}
\item  ~\hypertarget{glossary:pack}{\textbf{pack}} \\*
The process of copying data into a contiguous buffer.
(See also ~\hyperlink{glossary:unpack}{unpack}.)
Refer to Section~\ref{sec:pt2pt-packing} on page~\pageref{sec:pt2pt-packing}.

%%%%%%%%%%%%%%%%%%%%%%%%%%%%%%%%%%%%%%%%%%%%%%%%%%%%%%%%%%%%%%%%%%%%%%
\label{glossary:passive_target_communication}
\item  ~\hypertarget{glossary:passive_target_communication}{\textbf{passive target communication}} \\*
An \RMA/ communication where data is moved from the memory of one
process to the memory of another, and only the origin process is
explicitly involved
in
the transfer.  Thus, two origin processes may communicate by accessing
the same location in a target window.  The process that owns the
target window may be distinct from the two communicating processes, 
in which case it does not participate explicitly in the communication.
This communication
paradigm is closest to a shared memory model, where shared data can be
accessed by all processes, irrespective 
of location. (See also ~\hyperlink{glossary:active_target_communication}{active target communication} and
~\hyperlink{glossary:RMA}{RMA}.)
Refer to Section~\ref{sec:1sided-sync} on page~\pageref{sec:1sided-sync}.

%%%%%%%%%%%%%%%%%%%%%%%%%%%%%%%%%%%%%%%%%%%%%%%%%%%%%%%%%%%%%%%%%%%%%%
\label{glossary:persistent_communication}
\item  ~\hypertarget{glossary:persistent_communication}{\textbf{persistent communication}} \\*
A type of communication that binds the list of communication arguments to a {\bf persistent} communication
request once and, then, repeatedly uses
the request to initiate and complete messages.
For example, this is useful when
a communication pattern with the same argument list is repeatedly
executed within the inner loop of a parallel computation.   The
persistent request thus created can be thought of as a
communication port or a ``half-channel.''
It does not provide the full functionality of a conventional channel,
since there is no binding of the send port to the receive port. This
construct allows reduction of the overhead for communication
between the process and communication controller, but not of the overhead for
communication between one communication controller and another.
It is not necessary that messages sent with a persistent request be received
by a receive operation using a persistent request, or vice versa.
Refer to Section~\ref{sec:pt2pt-persistent} on page~\pageref{sec:pt2pt-persistent}.

%%%%%%%%%%%%%%%%%%%%%%%%%%%%%%%%%%%%%%%%%%%%%%%%%%%%%%%%%%%%%%%%%%%%%%
\label{glossary:PMPI}
\item  ~\hypertarget{glossary:PMPI}{\textbf{PMPI}} \\*
Profiling MPI interface. A name-shift interface that provides a mechanism to analyze MPI function usage.
Refer to Chapter~\ref{chap:prof} on page~\pageref{chap:prof}.

%%%%%%%%%%%%%%%%%%%%%%%%%%%%%%%%%%%%%%%%%%%%%%%%%%%%%%%%%%%%%%%%%%%%%%
\label{glossary:point-to-point communication}
\item  ~\hypertarget{glossary:point-to-point_communication}{\textbf{point-to-point communication}} \\*
Messages delivered from one sending process to one receiving process. (See also ~\hyperlink{glossary:collective}{collective}.)
Refer to Chapter~\ref{chap:pt2pt} on page~\pageref{chap:pt2pt}.

%%%%%%%%%%%%%%%%%%%%%%%%%%%%%%%%%%%%%%%%%%%%%%%%%%%%%%%%%%%%%%%%%%%%%%
\label{glossary:port_name}
\item  ~\hypertarget{glossary:port_name}{\textbf{port name}} \\*
A \mpiarg{port\_name} is a {\em system-supplied} string that encodes a
low-level network address at which a server can be
contacted. Typically this is an IP address and a port number, but an
implementation is free to use any protocol or naming convention.
(See also ~\hyperlink{glossary:service_name}{service name},
~\hyperlink{glossary:client}{client} and ~\hyperlink{glossary:server}{server}). 
Refer to Section~\ref{sec:client-server} on page~\pageref{sec:client-server}.

%%%%%%%%%%%%%%%%%%%%%%%%%%%%%%%%%%%%%%%%%%%%%%%%%%%%%%%%%%%%%%%%%%%%%%
\label{glossary:portable_datatype}
\item  ~\hypertarget{glossary:portable_datatype}{\textbf{portable datatype}} \\*
A datatype is portable if it is a predefined datatype or it is derived
from a portable datatype using only the type constructors
\mpifunc{MPI\_TYPE\_CONTIGUOUS}, \mpifunc{MPI\_TYPE\_VECTOR},
\mpifunc{MPI\_TYPE\_INDEXED},
% \mpifunc{MPI\_TYPE\_INDEXED\_BLOCK}, %% This name does not exist
\mpifunc{MPI\_TYPE\_CREATE\_INDEXED\_BLOCK},
\mpifunc{MPI\_TYPE\_CREATE\_SUBARRAY}, \mpifunc{MPI\_TYPE\_DUP}, and
\mpifunc{MPI\_TYPE\_CREATE\_DARRAY}.
Such a datatype is portable because all displacements in the datatype
are in terms of extents of one predefined datatype.  Therefore, if such a
datatype fits a data layout in one memory, it will fit the
corresponding data layout in another memory, if the same declarations
were used, even if the two systems have different architectures.  On
the other hand, if a datatype was constructed using
\mpifunc{MPI\_TYPE\_CREATE\_HINDEXED} or \mpifunc{MPI\_TYPE\_CREATE\_HVECTOR}, 
%% or \mpifunc{MPI\_TYPE\_CREATE\_STRUCT},      %% deprecated
then the datatype contains explicit byte
displacements (e.g., providing padding to meet alignment restrictions).
These displacements are unlikely to be chosen correctly if they fit
data layout on one memory, but are used for data layouts on another
process, running on a processor with a different architecture.
Refer to Section~\ref{terms:semantic} on page~\pageref{terms:semantic}.

%%%%%%%%%%%%%%%%%%%%%%%%%%%%%%%%%%%%%%%%%%%%%%%%%%%%%%%%%%%%%%%%%%%%%%
\label{glossary:predefined_datatype}
\item  ~\hypertarget{glossary:predefined_datatype}{\textbf{predefined datatype}} \\*
(See also ~\hyperlink{glossary:basic_datatype}{basic datatype} and
~\hyperlink{glossary:derived_datatype}{derived datatype}.)
Refer to Annex~\ref{chap:binding} on page~\pageref{chap:binding}.

%%%%%%%%%%%%%%%%%%%%%%%%%%%%%%%%%%%%%%%%%%%%%%%%%%%%%%%%%%%%%%%%%%%%%%
\label{glossary:process}
\item  ~\hypertarget{glossary:process}{\textbf{process}} \\*
An \MPI/ program consists of autonomous processes, executing their own
code, in an 
MIMD style. The codes executed by each process need not be
identical.  The processes communicate via calls to \MPI/ communication
primitives.  Typically, each process executes in its own address
space, although shared-memory implementations of \MPI/ are possible.
A process is represented in \MPI/ by a (group, rank) pair. 
A (group, rank) pair specifies a unique process but 
a process does not determine a unique (group, rank) pair, since
a process may belong to several groups.
Note that an \MPI/ process may or may not be correlated with
the operating system notion of a process.
Refer to Section~\ref{sec:macros} on page~\pageref{sec:macros}.

%%%%%%%%%%%%%%%%%%%%%%%%%%%%%%%%%%%%%%%%%%%%%%%%%%%%%%%%%%%%%%%%%%%%%%
\label{glossary:process_group}
\item  ~\hypertarget{glossary:process_group}{\textbf{process group}} \\*
See ~\hyperlink{glossary:groups}{groups}.
Refer to Section~\ref{subsec:pt2pt-envelope} on page~\pageref{subsec:pt2pt-envelope}, and
Section~\ref{sec:context} on page~\pageref{sec:context}.

%%%%%%%%%%%%%%%%%%%%%%%%%%%%%%%%%%%%%%%%%%%%%%%%%%%%%%%%%%%%%%%%%%%%%%
\label{glossary:progress}
\item  ~\hypertarget{glossary:progress}{\textbf{progress}} \\*
The property of parallel and distributed systems that
posted operations must complete in finite time
unless an exceptional condition (such as resource exhaustion)
causes an error.
Refer to point-to-point progress in Section~\ref{pt2pt-exE} on page~\pageref{pt2pt-exE},
file I/O progress in Section ~\ref{sec:io-intro-progress} on page ~\pageref{sec:io-intro-progress},
and RMA progress in Section ~\ref{sec:1sided-semantics} on page ~\pageref{sec:1sided-semantics}).

%%%%%%%%%%%%%%%%%%%%%%%%%%%%%%%%%%%%%%%%%%%%%%%%%%%%%%%%%%%%%%%%%%%%%%
\label{glossary:ready_communication_mode}
\item  ~\hypertarget{glossary:ready_communication_mode}{\textbf{ready communication mode}} \\*
A point-to-point communication protocol in which the communication
may be started {\em only} if the matching receive is already posted.
Otherwise, the operation is erroneous and its outcome is undefined.
On some systems, this allows the removal of a hand-shake
operation that is otherwise required and results in improved
performance.
The completion of the send operation does not depend on the
status of a matching receive, and merely indicates that the send
buffer can be reused.   A send operation that uses the ready mode has
the same semantics as a standard send operation, or a synchronous send
operation; it is merely that the sender provides additional
information to the system (namely that a matching receive is already
posted), that can save some overhead.  In a correct program, therefore, a
ready send could be replaced by a standard send with no effect on the
behavior of the program other than performance.
(See also ~\hyperlink{glossary:standard_communication_mode}{standard communication mode}, 
~\hyperlink{glossary:buffered_communication_mode}{buffered communication mode},
~\hyperlink{glossary:synchronous_communication_mode}{synchronous communication mode}.)
Refer to Section~\ref{sec:pt2pt-modes} on page~\pageref{sec:pt2pt-modes}.

%%%%%%%%%%%%%%%%%%%%%%%%%%%%%%%%%%%%%%%%%%%%%%%%%%%%%%%%%%%%%%%%%%%%%%
\label{glossary:reduce_operation}
\item  ~\hypertarget{glossary:reduce_operation}{\textbf{reduce operation}} \\*
A collective procedure in which an operation is performed on data supplied by the members of a group.
The reduction operation can be either one of a predefined list of
operations or a user-defined operation.
The global reduction functions come in several variations: a reduce that
returns the result of the reduction
%at one node,
to one member of a group,
an all-reduce that
returns this result
%at all nodes,
to all members of a group,
% a scan (parallel prefix) operation.  
scan (parallel prefix) operations,
non-blocking reductions,
and certain RMA accumulate operations.  
Refer to Section~\ref{global-reduce} on page~\pageref{global-reduce}.

%%%%%%%%%%%%%%%%%%%%%%%%%%%%%%%%%%%%%%%%%%%%%%%%%%%%%%%%%%%%%%%%%%%%%%
\label{glossary:remote_group}
\item  ~\hypertarget{glossary:remote_group}{\textbf{remote group}} \\*
The destinations of a procedure for an intercommunicator. 
(See also ~\hyperlink{glossary:intercommunicator}{intercommunicator} and
~\hyperlink{glossary:local_group}{local group}.)
Refer to Section~\ref{sec:coll-communicator} on page~\pageref{sec:coll-communicator}.

%%%%%%%%%%%%%%%%%%%%%%%%%%%%%%%%%%%%%%%%%%%%%%%%%%%%%%%%%%%%%%%%%%%%%%
\label{glossary:representation_conversion}
\item  ~\hypertarget{glossary:representation_conversion}{\textbf{type conversion}} \\*
Changing the binary representation of a value,
e.g., from Hex floating point to IEEE floating point. Note that the buffer size required for the receive can be affected by data conversions and
by the stride and other datatype layout factors of the receive datatype.  
(See also ~\hyperlink{glossary:representation_conversion}{representation conversion}.)
Refer to Section~\ref{subsec:pt2pt-conversion} on page~\pageref{subsec:pt2pt-conversion}.

%%%%%%%%%%%%%%%%%%%%%%%%%%%%%%%%%%%%%%%%%%%%%%%%%%%%%%%%%%%%%%%%%%%%%%
\label{glossary:RMA}
\item  ~\hypertarget{glossary:RMA}{\textbf{RMA}} \\*
Remote Memory Access (\RMA/) extends the communication mechanisms of \MPI/ by
allowing one process to specify all communication parameters, both for
the sending side and for the receiving side.
In some systems, message-passing and remote-memory-access (\RMA/) operations
run faster when accessing specially allocated memory (e.g., memory that is
shared by the other processes in the communicating group on an SMP).  \MPI/
provides a mechanism for allocating and freeing such special memory.  
(See also ~\hyperlink{glossary:active_target}{active target},
~\hyperlink{glossary:passive_target}{passive target},
~\hyperlink{glossary:window}{window},
~\hyperlink{glossary:access_epoch}{access epoch},
~\hyperlink{glossary:exposure_epoch}{exposure epoch},
~\hyperlink{glossary:origin_process}{origin process},
and ~\hyperlink{glossary:target_process}{target process}.)
Refer to Chapter~\ref{chap:one-side-2} on page~\pageref{chap:one-side-2}
and Section~\ref{sec:misc-memalloc} on page~\pageref{sec:misc-memalloc}).

%%%%%%%%%%%%%%%%%%%%%%%%%%%%%%%%%%%%%%%%%%%%%%%%%%%%%%%%%%%%%%%%%%%%%%
\label{glossary:root}
\item  ~\hypertarget{glossary:root}{\textbf{root}} \\*
The single process that originates or receives communication for those collective
operations that originate or receive to one process (e.g., broadcast and gather). 
(See also ~\hyperlink{glossary:collective_communication}{collective communication}.)
Refer to Section~\ref{sec:coll-intro} on page~\pageref{sec:coll-intro}.

%%%%%%%%%%%%%%%%%%%%%%%%%%%%%%%%%%%%%%%%%%%%%%%%%%%%%%%%%%%%%%%%%%%%%%
\label{glossary:scan_operation}
\item  ~\hypertarget{glossary:scan_operation}{\textbf{scan operation}} \\*
A collective procedure in which a partial reduction operation is performed on data supplied by the members of a group.
In the inclusive scan (see \func{MPI\_SCAN}), the prefix reduction on process i includes the data from process i. 
(See also ~\hyperlink{glossary:exscan_operation}{exscan operation} and
~\hyperlink{glossary:reduce_operation}{reduce operation}.)
Refer to Section~\ref{global-reduce} on page~\pageref{global-reduce}.

%%%%%%%%%%%%%%%%%%%%%%%%%%%%%%%%%%%%%%%%%%%%%%%%%%%%%%%%%%%%%%%%%%%%%%
\label{glossary:scope}
\item  ~\hypertarget{glossary:scope}{\textbf{scope}} \\*
The domain over which the \mpiarg{service\_name} 
can be retrieved when establishing communication among
two sets of \MPI/ processes that do not share a communicator.
[Refer to Section~\ref{sec:client-server} on page~\pageref{sec:client-server}.

%%%%%%%%%%%%%%%%%%%%%%%%%%%%%%%%%%%%%%%%%%%%%%%%%%%%%%%%%%%%%%%%%%%%%%
\label{glossary:send-receive_operation}
\item  ~\hypertarget{glossary:send-receive_operation}{\textbf{send-receive operation}} \\*
Operations that combine in one call the sending of a
message to one destination and the receiving of another message, from
another process.  The two (source and destination) may be the same.
A send-receive operation is
useful for executing a shift operation across a chain of
processes.  If blocking sends and receives are used for such a shift,
then one needs to order the sends and receives correctly (for
example, even processes
send, then receive, odd processes receive first, then send) so as to prevent
cyclic dependencies that may lead to deadlock.  When a send-receive
operation is used, the communication subsystem takes care of
these ordering issues.
Refer to Section~\ref{sec:pt2pt-sendrecv} on page~\pageref{sec:pt2pt-sendrecv}.

%%%%%%%%%%%%%%%%%%%%%%%%%%%%%%%%%%%%%%%%%%%%%%%%%%%%%%%%%%%%%%%%%%%%%%
\label{glossary:sequential_storage}
\item  ~\hypertarget{glossary:sequential_storage}{\textbf{sequential storage}} \\*
Variables that belong to the same array,
to the same {\sf COMMON} block in Fortran, or to the same structure in C.
Refer to Section~\ref{subsec:pt2pt-segmented} on page~\pageref{subsec:pt2pt-segmented}.

%%%%%%%%%%%%%%%%%%%%%%%%%%%%%%%%%%%%%%%%%%%%%%%%%%%%%%%%%%%%%%%%%%%%%%
\label{glossary:serialized}
\item  ~\hypertarget{glossary:serialized}{\textbf{serialized}} \\*
MPI calls are not made concurrently from two distinct threads in same process (all MPI calls are \emph{serialized}).
Refer to Section~\ref{sec:ei-threads} on page~\pageref{sec:ei-threads}.

%%%%%%%%%%%%%%%%%%%%%%%%%%%%%%%%%%%%%%%%%%%%%%%%%%%%%%%%%%%%%%%%%%%%%%
\label{glossary:server}
\item  ~\hypertarget{glossary:server}{\textbf{server}} \\*
The member of a client/server relationship that accepts requests.
 \MPI/ provides a mechanism for two sets of \MPI/  processes that do not share a communicator
to establish communication.
Establishing contact between two groups of processes that do not share an
existing communicator is a collective but asymmetric process.  One group of
processes indicates its willingness to accept connections from other groups of
processes.  This group is called the (parallel) \emph{server}, even if this
is not a client/server type of application.  
(See also ~\hyperlink{glossary:client}{client}, ~\hyperlink{glossary:port_name}{port name},
and ~\hyperlink{glossary:service_name}{service name}.) 
Refer to Section~\ref{sec:client-server} on page~\pageref{sec:client-server}.

%%%%%%%%%%%%%%%%%%%%%%%%%%%%%%%%%%%%%%%%%%%%%%%%%%%%%%%%%%%%%%%%%%%%%%
\label{glossary:service_name}
\item  ~\hypertarget{glossary:service_name}{\textbf{service name}} \\*
A \mpiarg{service\_name} is an {\em application-supplied} string for 
establishing communication between two independently running \MPI/ applications.
A service name can be published so that it
can be used by a client to connect to a server without knowing the  \mpiarg{port\_name}.
(See also ~\hyperlink{glossary:port_name}{port name}). 
Refer to Section~\ref{sec:client-server} on page~\pageref{sec:client-server}.

%%%%%%%%%%%%%%%%%%%%%%%%%%%%%%%%%%%%%%%%%%%%%%%%%%%%%%%%%%%%%%%%%%%%%%
\label{glossary:split_collective}
\item  ~\hypertarget{glossary:split_collective}{\textbf{split collective}} \\*
        \textbf{begin discussion ~\textgreater{}---------------------} \\*
        \emph{ this may be deprecated in mpi-3 } \\
        \textbf{---------------------~\textless{} end discussion}  \\
A restricted form of ``nonblocking'' operations
for collective file data access.
These routines are referred to as ``split'' collective routines
because a single collective operation is split in two:
a begin routine and an end routine.
The begin routine begins the operation,
much like a nonblocking data access (e.g., \func{MPI\_FILE\_IREAD}).
The end routine completes the operation,
much like the matching test or wait (e.g., \func{MPI\_WAIT}).
As with nonblocking data access operations,
the user must not use the buffer
% the user must not change the parameters
passed to a begin routine while the routine is outstanding;
the operation must be completed with an end routine before it
is safe to free buffers, etc.
Refer to Section~\ref{sec:io-split-collective} on page~\pageref{sec:io-split-collective}.

%%%%%%%%%%%%%%%%%%%%%%%%%%%%%%%%%%%%%%%%%%%%%%%%%%%%%%%%%%%%%%%%%%%%%%
\label{glossary:standard_communication_mode}
\item  ~\hypertarget{glossary:standard_communication_mode}{\textbf{standard communication mode}} \\*
A point-to-point communication protocol that leaves it up to \MPI/ to decide whether outgoing
messages will be buffered.  \MPI/ may
buffer outgoing messages.  In such a case, the send call may complete
before a matching receive is invoked.  On the other hand, buffer space may be
unavailable, or \MPI/ may choose not to buffer
outgoing messages, for performance reasons. In this case,
the send call will not complete until a matching receive has been posted, and
the data has been moved to the receiver.
Thus, a send in standard mode can be started whether or not a
matching receive has been posted.  It may complete before a matching receive
is posted.  The
standard mode send is {\bf non-local}:  successful completion of the send
operation may depend on the occurrence of a matching receive. 
(See also ~\hyperlink{glossary:buffered_communication_mode}{buffered communication mode}, 
~\hyperlink{glossary:synchronous_communication_mode}{synchronous communication mode},
~\hyperlink{glossary:ready_communication_mode}{ready communication mode}.)
Refer to Section~\ref{sec:pt2pt-modes} on page~\pageref{sec:pt2pt-modes}.

%%%%%%%%%%%%%%%%%%%%%%%%%%%%%%%%%%%%%%%%%%%%%%%%%%%%%%%%%%%%%%%%%%%%%%
\label{glossary:synchronizing_collective_communication}
\item  ~\hypertarget{glossary:synchronizing_collective_communication}{\textbf{synchronizing collective communication}} \\*
Communication between \MPI/ processes with the effect of constraining the relative temporal
order that those \MPI/  processes execute code. For example, \func{MPI\_BARRIER} 
blocks the caller until all group members have called it. The call returns at any process 
only after all group members have
entered the call.
Refer to Section~\ref{sec:coll-intro} on page~\pageref{sec:coll-intro}.

%%%%%%%%%%%%%%%%%%%%%%%%%%%%%%%%%%%%%%%%%%%%%%%%%%%%%%%%%%%%%%%%%%%%%%
\label{glossary:synchronous_communication_mode}
\item  ~\hypertarget{glossary:synchronous_communication_mode}{\textbf{synchronous communication mode}} \\*
A point-to-point communication mode in which the
communication can be started whether or
not a matching receive was posted.  However, the send will only complete 
successfully once a matching receive is posted, and the
receive operation has started to receive the message sent by the
synchronous send.
Thus, the completion of a synchronous send not only indicates that the
send buffer can be reused, but 
it also indicates that the receiver has
reached a certain point in its execution, namely that it has started
executing the matching receive.  If both sends and receives are
blocking operations then the use of the synchronous mode provides
synchronous communication semantics: a communication does not complete
at either end before both processes rendezvous at the
communication.  A send executed in this mode is non-local.
(See also ~\hyperlink{glossary:standard_communication_mode}{standard communication mode}, 
~\hyperlink{glossary:buffered_communication_mode}{buffered communication mode},
~\hyperlink{glossary:ready communication_mode}{ready communication mode}.)
Refer to Section~\ref{sec:pt2pt-modes} on page~\pageref{sec:pt2pt-modes}.

%%%%%%%%%%%%%%%%%%%%%%%%%%%%%%%%%%%%%%%%%%%%%%%%%%%%%%%%%%%%%%%%%%%%%%
\label{glossary:tag}
\item  ~\hypertarget{glossary:tag}{\textbf{tag}} \\*
An integer in point-to-point communication included as part of the \emph{message envelope}.
This integer can be used by the program to distinguish different types of
messages.
The range of valid tag values is {\sf 0,...,UB}, where the value of {\sf UB} is
implementation dependent.
(See also ~\hyperlink{glossary:message_envelope}{message envelope}.)
Refer to Section~\ref{subsec:pt2pt-envelope} on page~\pageref{subsec:pt2pt-envelope}.

%%%%%%%%%%%%%%%%%%%%%%%%%%%%%%%%%%%%%%%%%%%%%%%%%%%%%%%%%%%%%%%%%%%%%%
\label{glossary:target_process}
\item  ~\hypertarget{glossary:target_process}{\textbf{target process}} \\*
The process in which the memory is accessed in one-sided communications.
(See also ~\hyperlink{glossary:origin_process}{origin process} and 
~\hyperlink{glossary:RMA}{RMA}.)
Refer to Section~\ref{sec:one-side-2} on page~\pageref{sec:one-side-2}.

%%%%%%%%%%%%%%%%%%%%%%%%%%%%%%%%%%%%%%%%%%%%%%%%%%%%%%%%%%%%%%%%%%%%%%
\label{glossary:thread_compliant_implementation}
\item  ~\hypertarget{glossary:thread_compliant_implementation}{\textbf{thread compliant implementation}} \\*
An MPI process that may be multithreaded.
(See also ~\hyperlink{glossary:thread_safe}{thread safe} and ~\hyperlink{glossary:serialized}{serialized}.)
Refer to Section~\ref{sec:ei-threads} on page~\pageref{sec:ei-threads}.

%%%%%%%%%%%%%%%%%%%%%%%%%%%%%%%%%%%%%%%%%%%%%%%%%%%%%%%%%%%%%%%%%%%%%%
\label{glossary:thread_safe}
\item  ~\hypertarget{glossary:thread_safe}{\textbf{thread safe}} \\*
The property that two concurrently
running threads may make \MPI/ calls and the outcome will be as if the
calls executed in some order, even if their execution is interleaved.
(See also ~\hyperlink{glossary:thread_compliant_implementation}{thread compliant implementation}.)
Refer to Section~\ref{sec:ei-threads} on page~\pageref{sec:ei-threads}.

%%%%%%%%%%%%%%%%%%%%%%%%%%%%%%%%%%%%%%%%%%%%%%%%%%%%%%%%%%%%%%%%%%%%%%
\label{glossary:topology}
\item  ~\hypertarget{glossary:topology}{\textbf{topology}} \\*
The layout of processes within a virtual structure such as two or three
dimensional grid.
A topology is an extra,
optional attribute that one can give to an intra-communicator; topologies
cannot be added to inter-communicators.  A topology can provide a convenient
naming mechanism for the processes of a group (within a communicator), and
additionally, may assist the runtime system in mapping the processes onto
hardware.
(See also ~\hyperlink{glossary:groups}{groups},
~\hyperlink{glossary:cartesian_topology}{cartesian topology}, 
and ~\hyperlink{glossary:graph_topology}{graph topology}.)
Refer to Section~\ref{sec:topol} on page~\pageref{sec:topol}.

%%%%%%%%%%%%%%%%%%%%%%%%%%%%%%%%%%%%%%%%%%%%%%%%%%%%%%%%%%%%%%%%%%%%%%
\label{glossary:true_extent}
\item  ~\hypertarget{glossary:true_extent}{\textbf{true extent}} \\*
The true size of a datatype, i.e., the extent of the corresponding typemap, ignoring
\consti{MPI\_LB} and \mpiarg{MPI\_UB} markers, and performing no
rounding for alignment.  The \mpiarg{true\_extent} is the minimum number of bytes of
memory necessary to hold a datatype. 
If the typemap associated with
\mpiarg{datatype} is
\[
Typemap = \{ (type_0, disp_0), \ldots , (type_{n-1}, disp_{n-1})\}
\]
Then
\[
true\_lb(Typemap) = min_j  \{ disp_j ~:~ type_j \ne {\bf lb, ub} \},
\]
\[
true\_ub (Typemap) = max_j \{disp_j + sizeof(type_j) ~:~ type_j \ne
{\bf lb, ub}\} ,
\]
and
\[
true\_extent (Typemap) = true\_ub(Typemap) - true\_lb(typemap).
\]
(See also ~\hyperlink{glossary:lower_bound_datatype}{lower bound datatype}, ~\hyperlink{glossary:upper_bound_datatype}{upper bound datatype}, and ~\hyperlink{glossary:extent}{extent}.)
Refer to Section~\ref{sec:pt2pt-datatype} on page~\pageref{sec:pt2pt-datatype}.

%%%%%%%%%%%%%%%%%%%%%%%%%%%%%%%%%%%%%%%%%%%%%%%%%%%%%%%%%%%%%%%%%%%%%%
\label{glossary:type_conversion}
\item  ~\hypertarget{glossary:type_conversion}{\textbf{type conversion}} \\*
Changing the datatype of a value, e.g., by rounding a
\ftype{REAL} to an \ftype{INTEGER}.  
(See also ~\hyperlink{glossary:representation_conversion}{representation conversion}.)
Refer to Section~\ref{subsec:pt2pt-conversion} on page~\pageref{subsec:pt2pt-conversion}.


%%%%%%%%%%%%%%%%%%%%%%%%%%%%%%%%%%%%%%%%%%%%%%%%%%%%%%%%%%%%%%%%%%%%%%
\label{glossary:type_map}
\item  ~\hypertarget{glossary:type_map}{\textbf{type map}} \\*
The pair of sequences (or sequence of pairs) associated with a general dataype.
The displacements are not required to be positive, distinct, or
in increasing order. Therefore, the order of items need not
coincide with their order in store, and an item may appear more than
once. Type maps take the form
\[
Typemap = \{ (type_0,disp_0), ..., (type_{n-1}, disp_{n-1}) \} ,
\]
where $type_i$ are basic types, and
$disp_i$ are  displacements.
(See also ~\hyperlink{glossary:type_signature}{type signature},
~\hyperlink{glossary:basic_datatype}{basic datatype},
and ~\hyperlink{glossary:derived_datatype}{derived datatype}.)
Refer to Section~\ref{sec:pt2pt-datatype} on page~\pageref{sec:pt2pt-datatype}.

%%%%%%%%%%%%%%%%%%%%%%%%%%%%%%%%%%%%%%%%%%%%%%%%%%%%%%%%%%%%%%%%%%%%%%
\label{glossary:type_signature}
\item  ~\hypertarget{glossary:type_signature}{\textbf{type signature}} \\*
The sequences of types associated with a general dataype.
Type signatures may be used to validate matching types between sender and receiver; they take the form
\[
Typesig = \{ type_0 , ... , type_{n-1} \}
\]
where $type_i$ are basic datatypes.
(See also ~\hyperlink{glossary:type_map}{type map} and ~\hyperlink{glossary:basic_datatype}{basic datatype}.)
Refer to Section~\ref{sec:pt2pt-datatype} on page~\pageref{sec:pt2pt-datatype}.

%%%%%%%%%%%%%%%%%%%%%%%%%%%%%%%%%%%%%%%%%%%%%%%%%%%%%%%%%%%%%%%%%%%%%%
\label{glossary:unpack}
\item  ~\hypertarget{glossary:unpack}{\textbf{unpack}} \\*
The process of copying a contiguous buffer to a second buffer according to an \MPI/ datatype.
Note that an \MPI/ datatype may specify a non-contiguous structure.
(See also ~\hyperlink{glossary:pack}{pack}.)
Refer to Section~\ref{sec:pt2pt-packing} on page~\pageref{sec:pt2pt-packing}.

%%%%%%%%%%%%%%%%%%%%%%%%%%%%%%%%%%%%%%%%%%%%%%%%%%%%%%%%%%%%%%%%%%%%%%
\label{glossary:upper_bound_datatype}
\item  ~\hypertarget{glossary:upper_bound_datatype}{\textbf{upper bound, datatype}} \\*
The displacement of the highest unit of store which is addressed by the datatype. 
%%In general, if
%%\begin{displaymath}
%%Typemap = \{ (type_0 , disp_0 ) , ... , (type_{n-1} , disp_{n-1}) \} ,
%%\end{displaymath}
%%then the {\bf upper bound} of $Typemap$ is defined to be
%%\[
%%ub(Typemap) = \left\{ \begin{array}{ll}
%%\max_j disp_j + sizeof(type_j) + \epsilon & \mbox{if no entry has basic type
%%{\sf ub}}
%%\\ \max_j \{ disp_j \ \mbox{such that}\ type_j = {\sf ub} \} & \mbox{otherwise}
%%\end{array}
%%\right. \]
(See also ~\hyperlink{glossary:lower_bound}{lower bound} and ~\hyperlink{glossary:extent}{extent}.)
Refer to ub(Typemap) in Section~\ref{pt2pt-exX} on page~\pageref{pt2pt-exX}.

%%%%%%%%%%%%%%%%%%%%%%%%%%%%%%%%%%%%%%%%%%%%%%%%%%%%%%%%%%%%%%%%%%%%%%
\label{glossary:view_file}
\item  ~\hypertarget{glossary:view_file}{\textbf{view, file}} \\*
A {\it view} defines the current set of data visible
and accessible from an open file as an ordered set of etypes.
Each process has its own view of the file,
defined by three quantities:
a displacement, an etype, and a filetype.
The pattern described by a filetype is repeated,
beginning at the displacement, to define the view.
The pattern of repetition is defined to be the same pattern
that \func{MPI\_TYPE\_CONTIGUOUS} would produce if it were passed
the filetype and an arbitrarily large count.
%%Figure~\ref{fig:glossary-io-filetype} shows how the tiling works; note 
%%that the filetype in this example must have explicit 
%%lower and upper bounds set in order for the initial and final holes to be
%%repeated in the view.
%%Views can be changed by the user during program execution.
%%The default view is a linear byte stream
%%(displacement is zero, etype and filetype equal to \type{MPI\_BYTE}).
%%
%%\begin{figure}[htpb]
%%  \centerline{\includegraphics[width=4.0in]{figures/io-filetype}}
%%  \caption{Etypes and filetypes}
%%  \label{fig:glossary-io-filetype}
%%\end{figure}
%%
%%A group of processes can use complementary views to
%%achieve a global data distribution such as a scatter/gather pattern
%%(see Figure~\ref{fig:gloss-io-comp-filetypes}).
%%
%%\begin{figure}[htpb]
%%  \centerline{\includegraphics[width=4.0in]{figures/io-comp-filetypes}}
%%  \caption{Partitioning a file among parallel processes}
%%  \label{fig:gloss-io-comp-filetypes}
%%\end{figure}
Refer to Section~\ref{subsec:io-2:definitions} on page~\pageref{subsec:io-2:definitions}.

%%%%%%%%%%%%%%%%%%%%%%%%%%%%%%%%%%%%%%%%%%%%%%%%%%%%%%%%%%%%%%%%%%%%%%
\label{glossary:wildcard}
\item  ~\hypertarget{glossary:wildcard}{\textbf{wildcard}} \\*
The receive of a point-to-point message may utilize
a special \emph{tag} (\const{MPI\_ANY\_TAG}) 
or \emph{source} (\const{MPI\_ANY\_SOURCE}) that
indicates any source and/or tag are acceptable.  Note that
wildcards may not be used to constrain \emph{communicators}.
Refer to Section~\ref{sec:one-side-2} on page~\pageref{sec:one-side-2}.

%%%%%%%%%%%%%%%%%%%%%%%%%%%%%%%%%%%%%%%%%%%%%%%%%%%%%%%%%%%%%%%%%%%%%%
\label{glossary:window}
\item  ~\hypertarget{glossary:window}{\textbf{window}} \\*
A range of memory that is made accessible to accesses by remote 
processes using one sided communication. 
In one-sided communications, each process specifies
a window of existing memory that it exposes to \RMA/ accesses by the
processes in the group of  \mpiarg{comm}.
The window consists of \mpiarg{size} bytes,
starting at address \mpiarg{base}.
(See also ~\hyperlink{glossary:RMA}{RMA}.)
Refer to Section~\ref{sec:one-side-2} on page~\pageref{sec:one-side-2}.


\end{itemize}