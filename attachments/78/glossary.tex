% glossary.tex
%
%
%
% Version as of Aug-04-2011
% Edited by Terry Jones
%
%
%
%%\status{pre First Reading}

\chapter{Glossary}
\label{glossary}
\begin{itemize}

%%%%%%%%%%%%%%%%%%%%%%%%%%%%%%%%%%%%%%%%%%%%%%%%%%%%%%%%%%%%%%%%%%%%%%
\label{glossary:absolute_address}
\item \bf{ absolute address} \\*
Displacements relative to ``address
zero,'' the start of the address space.
See Section~\ref{pt2pt-addfunc} on page~\pageref{pt2pt-addfunc}.

%%%%%%%%%%%%%%%%%%%%%%%%%%%%%%%%%%%%%%%%%%%%%%%%%%%%%%%%%%%%%%%%%%%%%%
\label{glossary:access_epic}
\item \bf{ access epic} \\*
The period during which a target window can be accessed by \RMA/
operations.
(See also ~\ref{glossary:exposure_epoch}.)
See Section~\ref{sec:1sided-sync} on page~\pageref{sec:1sided-sync}.

%%%%%%%%%%%%%%%%%%%%%%%%%%%%%%%%%%%%%%%%%%%%%%%%%%%%%%%%%%%%%%%%%%%%%%
\label{glossary:active}
\item \bf{ active} \\* 
We say that a parallel procedure is {\em active} in a process if the process
belongs to a group that may collectively execute the procedure, and
some member of that group is currently executing the procedure code.
If a parallel procedure is active in a process, then this process may
be receiving messages pertaining to this procedure, even if it
does not currently execute the code of this procedure.
See Section~\ref{sec:formalizing} on page~\pageref{sec:formalizing}.

%%%%%%%%%%%%%%%%%%%%%%%%%%%%%%%%%%%%%%%%%%%%%%%%%%%%%%%%%%%%%%%%%%%%%%
\label{glossary:active_target}
\item \bf{ active target} \\*
An \RMA/ communication where data is moved from the memory of one
process to the memory of another, and both are explicitly involved in the
communication.  This communication pattern is similar to message
passing, except that all the data transfer arguments are provided by
one process, and the second process only participates in the synchronization.
(See also ~\ref{glossary:passive_target}.)
See Section~\ref{sec:1sided-sync} on page~\pageref{sec:1sided-sync}.

%%%%%%%%%%%%%%%%%%%%%%%%%%%%%%%%%%%%%%%%%%%%%%%%%%%%%%%%%%%%%%%%%%%%%%
\label{glossary:associative}
\item \bf{ type associative} \\*
The property of a collective reduction, namely that the order in which the operations are 
performed does not matter as long as the sequence of the operands is not changed. 
(See also ~\ref{glossary:commutative}.)
See Section~\ref{subsec:coll-user-ops} on page~\pageref{subsec:coll-user-ops}.

%%%%%%%%%%%%%%%%%%%%%%%%%%%%%%%%%%%%%%%%%%%%%%%%%%%%%%%%%%%%%%%%%%%%%%
\label{glossary:blocking}
\item \bf{ blocking} \\*
A procedure is blocking if return from the procedure indicates the user
is allowed to reuse resources specified in the call.
\item \bf{ local}]
A procedure is local if completion of the procedure depends only on the
local executing process.
See Section~\ref{terms:semantic} on page~\pageref{terms:semantic}.

%%%%%%%%%%%%%%%%%%%%%%%%%%%%%%%%%%%%%%%%%%%%%%%%%%%%%%%%%%%%%%%%%%%%%%
\label{glossary:broadcast}
\item \bf{ broadcast} \\*
A collective operation which communicates data from a root process to all processes;
initially just the first process contains the data, but after the
broadcast all processes contain it.
See Section~\ref{sec:coll-intro} on page~\pageref{sec:coll-intro}.

%%%%%%%%%%%%%%%%%%%%%%%%%%%%%%%%%%%%%%%%%%%%%%%%%%%%%%%%%%%%%%%%%%%%%%
\label{glossary:buffered_communication_mode}
\item \bf{ buffered communication mode} \\*
A communication protocol in which the
send operation can be started whether or not a
matching receive has been posted.
It may complete before a matching receive is posted.  However, unlike
the standard send, this operation is {\bf local}, and its
completion does not depend on the occurrence of a matching receive.  Thus, if a
send is executed and no matching receive is posted, then \MPI/ must buffer the
outgoing message, so as to allow the send call to complete.   An error will
occur if there is insufficient buffer space.   The amount of available buffer
space is controlled by the user.
Buffer allocation by the user may be required for the buffered mode to be
effective. 
(See also ~\ref{glossary:standard_communication_mode}, 
~\ref{glossary:synchronous_communication_mode},
~\ref{glossary:ready_communication_mode}.)
See Section~\ref{sec:pt2pt-modes} on page~\pageref{sec:pt2pt-modes}.

%%%%%%%%%%%%%%%%%%%%%%%%%%%%%%%%%%%%%%%%%%%%%%%%%%%%%%%%%%%%%%%%%%%%%%
\label{glossary:caching}
\item \bf{ caching} \\*
\MPI/ provides a ``caching'' facility that allows an application to
attach arbitrary pieces of information, called {\bf attributes}, to
% communicators.  More precisely, the caching
three kinds of MPI objects, communicators, windows and datatypes.
More precisely, the caching
facility allows a portable library to do the following:
\begin{itemize}
\item
  pass information between calls by associating it
  with an \MPI/ intra- or in\-ter-\-com\-mun\-i\-ca\-tor, 
window or datatype,
\item quickly retrieve that information, and
\item
 be guaranteed that out-of-date information is never retrieved, even if
 % the communicator is freed and its handle subsequently reused by \MPI/.
 the object is freed and its handle subsequently reused by \MPI/.
\end{itemize}
See Section~\ref{sec:caching} on page~\pageref{sec:caching}.

%%%%%%%%%%%%%%%%%%%%%%%%%%%%%%%%%%%%%%%%%%%%%%%%%%%%%%%%%%%%%%%%%%%%%%
\label{glossary:client}
\item \bf{ client} \\*
 \MPI/ provides a mechanism for two sets of \MPI/  processes that do not share a communicator
to establish communication.
Establishing contact between two groups of processes that do not share an
existing communicator is a collective but asymmetric process.  One group connects to the
server; we will call it the \emph{client}. 
(See also ~\ref{glossary:server}). 
See Section~\ref{sec:client-server} on page~\pageref{sec:client-server}.

%%%%%%%%%%%%%%%%%%%%%%%%%%%%%%%%%%%%%%%%%%%%%%%%%%%%%%%%%%%%%%%%%%%%%%
\label{glossary:collective}
\item \bf{ collective} \\*
A procedure is collective if all processes in a process group need to invoke the procedure.  A
collective call may or may not be synchronizing.
Collective calls over the same communicator
must be executed in the same order by all members of the process
group.
See Section~\ref{terms:semantic} on page~\pageref{terms:semantic}.

%%%%%%%%%%%%%%%%%%%%%%%%%%%%%%%%%%%%%%%%%%%%%%%%%%%%%%%%%%%%%%%%%%%%%%
\label{glossary:committed_datatype}
\item \bf{ committed datatype} \\*
The second step in preparing a datatype for communication (after ``created datatype'').
There is no need to commit basic datatypes. They are ``pre-committed.''
(See also ~\ref{glossary:standard_communication_mode}.)
See Section~\ref{subsec:pt2pt-comfree} on page~\pageref{subsec:pt2pt-comfree}.

%%%%%%%%%%%%%%%%%%%%%%%%%%%%%%%%%%%%%%%%%%%%%%%%%%%%%%%%%%%%%%%%%%%%%%
\label{glossary:communicator}
\item \bf{ communicator} \\*
A communicator specifies the
communication context for a communication operation.
Each communication context provides a separate ``communication
universe:'' messages are always received within the context they were
sent, and messages sent in different contexts do not interfere.
The communicator also specifies the set of processes that share this
communication context. 
See Section~\ref{subsec:pt2pt-envelope} on page~\pageref{subsec:pt2pt-envelope}.

%%%%%%%%%%%%%%%%%%%%%%%%%%%%%%%%%%%%%%%%%%%%%%%%%%%%%%%%%%%%%%%%%%%%%%
\label{glossary:communication context}
\item \bf{ communication context} \\*
See ~\ref{glossary:contexts}.

%%%%%%%%%%%%%%%%%%%%%%%%%%%%%%%%%%%%%%%%%%%%%%%%%%%%%%%%%%%%%%%%%%%%%%
\label{glossary:commutative}
\item \bf{ type commutative} \\*
The property of a collective reduction, namely that changing the order of the operands does not change the end result. 
(See also ~\ref{glossary:associative}.)
See Section~\ref{subsec:coll-user-ops} on page~\pageref{subsec:coll-user-ops}.

%%%%%%%%%%%%%%%%%%%%%%%%%%%%%%%%%%%%%%%%%%%%%%%%%%%%%%%%%%%%%%%%%%%%%%
\label{glossary:completion/completed}
\item \bf{ type completion/completed} \\*
The word complete is used with respect to operations, requests, and communications. 
An operation completes when the user is allowed to reuse resources, and any output 
buffers have been updated; i.e. a call to \func{MPI\_TEST} will return flag = true. A request is 
completed by a call to wait, which returns, or a test or get status call which returns flag = true. 
This completing call has two effects: the status is extracted from the request; in the case 
of test and wait, if the request was non persistent, it is freed, and becomes inactive if it 
was persistent. A communication completes when all participating operations complete.
See Section~\ref{terms:semantic} on page~\pageref{terms:semantic}.


%%%%%%%%%%%%%%%%%%%%%%%%%%%%%%%%%%%%%%%%%%%%%%%%%%%%%%%%%%%%%%%%%%%%%%
\label{glossary:contexts}
\item \bf{ contexts} \\*
Contexts provide the ability to have
a separate safe ``universe''
of message-passing between the two groups.  A send in the local
group is always a receive in the remote group, and vice versa.
The system manages this differentiation process.
The use of separate communication
contexts by distinct libraries (or distinct library invocations)
insulates communication internal to the library execution from
external communication.  This allows the invocation of the library even if
there are pending communications
on ``other'' communicators, and avoids the need to
synchronize entry or exit into library 
code.
See Section~\ref{sec:context} on page~\pageref{sec:context}.

%%%%%%%%%%%%%%%%%%%%%%%%%%%%%%%%%%%%%%%%%%%%%%%%%%%%%%%%%%%%%%%%%%%%%%
\label{glossary:contiguous}
\item \bf{ contiguous} \\*
A collection of memory locations that are adjacent to one another
without intervening extraneous data.
See Section~\ref{sec:coll-intro} on page~\pageref{sec:coll-intro}.

%%%%%%%%%%%%%%%%%%%%%%%%%%%%%%%%%%%%%%%%%%%%%%%%%%%%%%%%%%%%%%%%%%%%%%
\label{glossary:correct_program}
\item \bf{ correct program} \\*
A program that performs as intended; A program that is free of bugs. 
For example, a  {\bf correct program} must invoke collective communications so
that deadlock will
not occur, whether collective communications are synchronizing or not.
(See also ~\ref{glossary:erroneous_program}.)
See Section~\ref{coll:correct} on page~\pageref{coll:correct}.

%%%%%%%%%%%%%%%%%%%%%%%%%%%%%%%%%%%%%%%%%%%%%%%%%%%%%%%%%%%%%%%%%%%%%%
\label{glossary:created_datatype}
\item \bf{ created datatype} \\*
The initial step in preparing a datatype for communication (before ``committed datatype'').
(See also ~\ref{glossary:standard_communication_mode}.)
See Section~\ref{subsec:pt2pt-comfree} on page~\pageref{subsec:pt2pt-comfree}.

%%%%%%%%%%%%%%%%%%%%%%%%%%%%%%%%%%%%%%%%%%%%%%%%%%%%%%%%%%%%%%%%%%%%%%
\label{glossary:deprecated}
\item \bf{ deprecated} \\*
Constructs that continue to be part of the \MPI/ standard, 
as documented in Chapter~\ref{chap:deprecated}, 
but that users are recommended not to continue using, since 
better solutions were provided with \MPIII/.
For example, the Fortran binding 
for \mpii/ functions that have address arguments uses {\tt INTEGER}.
This is not consistent with the C binding, and causes problems on
machines with 32 bit {\tt INTEGER}s and 64 bit addresses.
See Section~\ref{sec:deprecated} on page~\pageref{sec:deprecated}.

%%%%%%%%%%%%%%%%%%%%%%%%%%%%%%%%%%%%%%%%%%%%%%%%%%%%%%%%%%%%%%%%%%%%%%
\label{glossary:derived_datatype}
\item \bf{ derived datatype} \\*
A derived datatype is any datatype that is not predefined.
(See also ~\ref{glossary:general_datatype} and ~\ref{glossary:derived_datatype} .)
See Section~\ref{terms:semantic} on page~\pageref{terms:semantic}.

%%%%%%%%%%%%%%%%%%%%%%%%%%%%%%%%%%%%%%%%%%%%%%%%%%%%%%%%%%%%%%%%%%%%%%
\label{glossary:displacement}
\item \bf{ displacement} \\*
A file {\it displacement} is an absolute byte position
relative to the beginning of a file.
The displacement defines the location where a {\it view} begins.
% A displacement is an offset relative to the default view (see below).
Note that a ``file displacement'' is distinct from a ``typemap displacement.''
See Section~\ref{subsec:io-2:definitions} on page~\pageref{subsec:io-2:definitions}.

%%%%%%%%%%%%%%%%%%%%%%%%%%%%%%%%%%%%%%%%%%%%%%%%%%%%%%%%%%%%%%%%%%%%%%
\label{glossary:equivalent}
\item \bf{ equivalent} \\*
Two datatypes are equivalent if they appear to have been created with
the same sequence of calls (and arguments) and thus have the same
typemap.  Two equivalent datatypes do not necessarily have the same
cached attributes or the same names.
See Section~\ref{terms:semantic} on page~\pageref{terms:semantic}.

%%%%%%%%%%%%%%%%%%%%%%%%%%%%%%%%%%%%%%%%%%%%%%%%%%%%%%%%%%%%%%%%%%%%%%
\label{glossary:erroneous_program}
\item \bf{ erroneous program} \\*
A program that does not perform as intended; A program that contains bugs. 
(See also ~\ref{glossary:correct_program}.)
See Section~\ref{coll:correct} on page~\pageref{coll:correct}.

%%%%%%%%%%%%%%%%%%%%%%%%%%%%%%%%%%%%%%%%%%%%%%%%%%%%%%%%%%%%%%%%%%%%%%
\label{glossary:error_class}
\item \bf{ error class} \\*
The error codes returned by \MPI/ are left entirely to the
implementation (with the
exception of \const{MPI\_SUCCESS}).  This is done to allow an implementation to
provide as much information as possible in the error code (for use with
\mpifunc{MPI\_ERROR\_STRING}).
To make it possible for an application to interpret an error code, the routine
\mpifunc{MPI\_ERROR\_CLASS} %mansplit
converts any error code into one of a small set of standard error
codes, called {\em error classes}.  
% Valid error classes include
Valid error classes are shown in Table~\ref{table:inquiry:errclasses:part:i}
and Table~\ref{table:inquiry:errclasses:part:ii}. 
See Section~\ref{table:inquiry:errclasses:part:i} on page~\pageref{table:inquiry:errclasses:part:i}.

%%%%%%%%%%%%%%%%%%%%%%%%%%%%%%%%%%%%%%%%%%%%%%%%%%%%%%%%%%%%%%%%%%%%%%
\label{glossary:etype}
\item \bf{ etype} \\* 
An {\it etype} ({\it elementary} datatype)
is the unit of data access and positioning.
It can be any \MPI/ predefined or derived datatype.
Derived etypes can be constructed
using any of the \MPI/ datatype constructor routines,
provided all resulting typemap displacements are non-negative
and monotonically nondecreasing.
Data access is performed in etype units,
reading or writing whole data items of type etype.
Offsets are expressed as a count of etypes;
file pointers point to the beginning of etypes.
Depending on context,
the term ``etype'' is used to describe one of three aspects
of an elementary datatype:
a particular \MPI/ type,
a data item of that type,
or the extent of that type.
See Section~\ref{subsec:io-2:definitions} on page~\pageref{subsec:io-2:definitions}.

%%%%%%%%%%%%%%%%%%%%%%%%%%%%%%%%%%%%%%%%%%%%%%%%%%%%%%%%%%%%%%%%%%%%%%
\label{glossary:exposure_epic}
\item \bf{ exposure epic} \\*
The period during which a target window can be accessed by \RMA/
operations in {\it active target} communication.
(See also ~\ref{glossary:access_epoch}.)
See Section~\ref{sec:1sided-sync} on page~\pageref{sec:1sided-sync}.

%%%%%%%%%%%%%%%%%%%%%%%%%%%%%%%%%%%%%%%%%%%%%%%%%%%%%%%%%%%%%%%%%%%%%%
\label{glossary:extent}
\item \bf{ extent} \\*
The {\bf extent} of a datatype is defined to
be the span from the first byte to the last byte occupied by entries in this
datatype, rounded up to satisfy alignment requirements.
That is, if
\[
Typemap = \{ (type_0,disp_0), ..., (type_{n-1}, disp_{n-1}) \} ,
\]
then
\begin{eqnarray}
lb(Typemap) & = & \min_j disp_j , \nonumber \\
ub(Typemap) & = & \max_j (disp_j + sizeof(type_j)) + \epsilon , \mbox{ and}
\nonumber \\ extent(Typemap) & = & ub(Typemap) -lb(Typemap).
\end{eqnarray}
If $type_i$ requires alignment to a byte address that 
% is is 
is
a multiple
of $k_i$,
then $\epsilon$ is the least non-negative increment needed to round
$extent(Typemap)$ to the next multiple of $\max_i k_i$.
For datatypes that 
have a ``hole'' at its beginning or its end, or a datatype with
entries that extend above the upper bound or below the lower bound, then
\label{eq:pt2pt-extent}
\[
extent(Typemap) = ub(Typemap) - lb(Typemap)
\]
(See also ~\ref{glossary:lower_bound}, ~\ref{glossary:upper_bound}, and ~\ref{glossary:true_extent}.)
See Section~\ref{sec:pt2pt-datatype} on page~\pageref{sec:pt2pt-datatype}.

%%%%%%%%%%%%%%%%%%%%%%%%%%%%%%%%%%%%%%%%%%%%%%%%%%%%%%%%%%%%%%%%%%%%%%
\label{glossary:external32}
\item \bf{ external32} \\*
Data in \emph{external32} data representation states that read and write operations
convert all data from
and to the ``external32''
representation defined in Section~\ref{subsec:ext32},
page~\pageref{subsec:ext32}.
The data conversion rules for communication also apply to these
conversions (see Section~3.3.2, page~25-27, of the \MPII/
document).
The data on the storage
medium is always in this canonical representation, and
the data in memory
is always in the local process's native representation.
% is always in the representation of the process that placed it there.
(See also ~\ref{glossary:native} and ~\ref{glossary:internal}.)
See Section~\ref{sec:io-file-interop} on page~\pageref{sec:io-file-interop}.

%%%%%%%%%%%%%%%%%%%%%%%%%%%%%%%%%%%%%%%%%%%%%%%%%%%%%%%%%%%%%%%%%%%%%%
\label{glossary:fairness}
\item \bf{ fairness} \\*
The property of parallel and distributed systems that no process is starved, 
and all processes are accorded the same priority in allowing their accesses 
to shared resources. When fairness is imposed, all processes have the 
chance to make progress regardless of what other processes may be 
doing at the same time. Note that \MPI/  makes no fairness guarantees.
Suppose that a send is posted.  Then it is possible
that the destination process repeatedly posts a receive that matches this
send, yet the message is never received, because it is each time overtaken by
another message, sent from another source.  Similarly, suppose that a
receive was posted by a multi-threaded process.  Then it is possible that
messages that
match this receive are repeatedly received, yet the receive is never satisfied,
because it is overtaken by other receives posted at this node (by
other executing threads).  It is the programmer's responsibility to prevent
starvation in such situations.
See Section~\ref{pt2pt-exF} on page~\pageref{pt2pt-exF}.

%%%%%%%%%%%%%%%%%%%%%%%%%%%%%%%%%%%%%%%%%%%%%%%%%%%%%%%%%%%%%%%%%%%%%%
\label{glossary:file}
\item \bf{ file} \\*
An \MPI/ file is an ordered collection of typed data items.
\MPI/ supports random or sequential access to any integral set of these items.
% An \MPI/ file is an ordered collection of data bytes
% supporting random or sequential access to any integral set of bytes.
A file is opened collectively by a group of processes.
All collective I/O calls on a file are collective over this group.
See Section~\ref{subsec:io-2:definitions} on page~\pageref{subsec:io-2:definitions}.

%%%%%%%%%%%%%%%%%%%%%%%%%%%%%%%%%%%%%%%%%%%%%%%%%%%%%%%%%%%%%%%%%%%%%%
\label{glossary:file_consistency}
\item \bf{ file consistency} \\*
Consistency semantics define the outcome of multiple accesses
to a single file.
All file accesses in \MPI/ are relative to a specific file handle
created from a collective open.
\MPI/ provides three levels of consistency:
sequential consistency among all accesses using a single file handle,
sequential consistency among all accesses
using file handles created from a single collective open
with atomic mode enabled,
and
user-imposed consistency among accesses other than the above.
%--via the \func{MPI\_FILE\_SYNC} routine.
%-weak consistency among accesses not handled above
%-via the \func{MPI\_FILE\_SYNC} routine.
Sequential consistency means the behavior of a set of operations
will be as if the operations were performed in some serial order
consistent with program order; each access appears atomic,
although the exact ordering of accesses is unspecified.
User-imposed consistency may be obtained using program order
and calls to \func{MPI\_FILE\_SYNC}.
See Section~\ref{io-filecntl-atomicity} on page~\pageref{io-filecntl-atomicity}.

%%%%%%%%%%%%%%%%%%%%%%%%%%%%%%%%%%%%%%%%%%%%%%%%%%%%%%%%%%%%%%%%%%%%%%
\label{glossary:file_handle}
\item \bf{ file handle} \\* 
A {\it file handle} is an opaque object created by \func{MPI\_FILE\_OPEN}
and freed by \func{MPI\_FILE\_CLOSE}.
All operations on an open file
reference the file through the file handle.
See Section~\ref{subsec:io-2:definitions} on page~\pageref{subsec:io-2:definitions}.

%%%%%%%%%%%%%%%%%%%%%%%%%%%%%%%%%%%%%%%%%%%%%%%%%%%%%%%%%%%%%%%%%%%%%%
\label{glossary:file_interoperability}
\item \bf{ file interoperability} \\* 
file interoperability is the ability to
read the information previously written to a file---not just the
bits of data, but the actual information the bits represent.
\MPI/ guarantees full interoperability within a single \MPI/ environment,
and supports increased interoperability outside that environment
through the external data representation (Section~\ref{subsec:ext32}, page~\pageref{subsec:ext32}) as
well as the data conversion functions (Section~\ref{sec:io-datarep},
page~\pageref{sec:io-datarep}). 
See Section~\ref{sec:io-file-interop} on page~\pageref{sec:io-file-interop}.

%%%%%%%%%%%%%%%%%%%%%%%%%%%%%%%%%%%%%%%%%%%%%%%%%%%%%%%%%%%%%%%%%%%%%%
\label{glossary:file_pointer}
\item \bf{ file pointer} \\*
A {\it file pointer} is an implicit offset maintained by \MPI/.
``Individual file pointers'' are file pointers that are local to
each process that opened the file.
A ``shared file pointer'' is a file pointer that is shared by
the group of processes that opened the file.
See Section~\ref{subsec:io-2:definitions} on page~\pageref{subsec:io-2:definitions}.

%%%%%%%%%%%%%%%%%%%%%%%%%%%%%%%%%%%%%%%%%%%%%%%%%%%%%%%%%%%%%%%%%%%%%%
\label{glossary:file_size}
\item \bf{ file size} \\*
The {\it size} of an \MPI/ file is measured in bytes from the 
beginning of the file.  A newly created file has a size of zero 
bytes.  Using the size as an absolute displacement gives 
the position of the byte immediately following the last byte in 
the file.  For any given view, the {\it end of file} is the 
offset of the first etype accessible in the current view starting
after the last byte in the file.
% The file size and the
% end of file may vary between different processes.
See Section~\ref{subsec:io-2:definitions} on page~\pageref{subsec:io-2:definitions}.

%%%%%%%%%%%%%%%%%%%%%%%%%%%%%%%%%%%%%%%%%%%%%%%%%%%%%%%%%%%%%%%%%%%%%%
\label{glossary:filetype}
\item \bf{ filetype} \\* 
A {\it filetype} is the basis for partitioning a file among processes
and defines a template for accessing the file.
A filetype is either a single etype or a derived \MPI/ datatype
constructed from multiple instances of the same etype.
In addition,
the extent of any hole in the filetype
must be a multiple of the etype's extent.
The displacements in the typemap of the filetype are not required to be distinct,
See Section~\ref{subsec:io-2:definitions} on page~\pageref{subsec:io-2:definitions}.

%%%%%%%%%%%%%%%%%%%%%%%%%%%%%%%%%%%%%%%%%%%%%%%%%%%%%%%%%%%%%%%%%%%%%%
\label{glossary:freed_datatype}
\item \bf{ freed datatype} \\*
To mark a datatype object associated with \mpiarg{datatype} for
deallocation and set the \mpiarg{datatype} to \const{MPI\_DATATYPE\_NULL}.
Any communication that is currently using this datatype will complete 
normally.
See Section~\ref{subsec:pt2pt-comfree} on page~\pageref{subsec:pt2pt-comfree}.

%%%%%%%%%%%%%%%%%%%%%%%%%%%%%%%%%%%%%%%%%%%%%%%%%%%%%%%%%%%%%%%%%%%%%%
\label{glossary:gather}
\item \bf{ gather} \\*
A collective operation in which {\tt n} messages sent by the
processes in the group are concatenated in rank order, and the
resulting message is received by the root as if by a call to
\mpifunc{MPI\_RECV(recvbuf, recvcount$\cdot$n, recvtype, ...)}.
See Section~\ref{sec:coll-intro} on page~\pageref{sec:coll-intro}.

%%%%%%%%%%%%%%%%%%%%%%%%%%%%%%%%%%%%%%%%%%%%%%%%%%%%%%%%%%%%%%%%%%%%%%
\label{glossary:general_datatype}
\item \bf{ general datatype} \\*
A {\bf general datatype} is an opaque object that specifies two
things:
\begin{itemize}
\item
A sequence of basic datatypes
\item
A sequence of integer (byte) displacements
\end{itemize}
The displacements are not required to be positive, distinct, or
in increasing order. Therefore, the order of items need not
coincide with their order in store, and an item may appear more than
once. (See also ~\ref{glossary:predefined_datatype} and ~\ref{glossary:derived_datatype} .)
See Section~\ref{terms:semantic} on page~\pageref{terms:semantic}.

%%%%%%%%%%%%%%%%%%%%%%%%%%%%%%%%%%%%%%%%%%%%%%%%%%%%%%%%%%%%%%%%%%%%%%
\label{glossary:generalized_request}
\item \bf{ generalized request} \\*
A user defined non-blocking operation.
See Section~\ref{sec:ei-intro} on page~\pageref{sec:ei-intro}.

%%%%%%%%%%%%%%%%%%%%%%%%%%%%%%%%%%%%%%%%%%%%%%%%%%%%%%%%%%%%%%%%%%%%%%
\label{glossary:global}
\item \bf{ global} \\*
Referring to all members of a group.
See Section~\ref{sec:coll-intro} on page~\pageref{sec:coll-intro}.

%%%%%%%%%%%%%%%%%%%%%%%%%%%%%%%%%%%%%%%%%%%%%%%%%%%%%%%%%%%%%%%%%%%%%%
\label{glossary:groups}
\item \bf{ groups} \\*
Groups define an ordered collection of processes, each with a rank, and it is this
group that defines the low-level names for inter-process communication (ranks
are used for sending and receiving).  Thus, groups define a scope for process
names in point-to-point communication.  In addition, groups define the scope
of collective operations.  Groups may be manipulated separately from
communicators in \MPI/, but only communicators can be used in
communication operations. Each process in
the group is assigned a rank between {\tt 0} and {\tt n-1}.
(See also ~\ref{glossary:topology} .)
See Section~\ref{sec:context} on page~\pageref{sec:context}.

%%%%%%%%%%%%%%%%%%%%%%%%%%%%%%%%%%%%%%%%%%%%%%%%%%%%%%%%%%%%%%%%%%%%%%
\label{glossary:implementation}
\item \bf{ implementation} \\*
A specific fulfillment of a specification.  
See Section~\ref{terms:semantic} on page~\pageref{terms:semantic}.

%%%%%%%%%%%%%%%%%%%%%%%%%%%%%%%%%%%%%%%%%%%%%%%%%%%%%%%%%%%%%%%%%%%%%%
\label{glossary:IN}
\item \bf{ IN} \\*
An argument of an \MPI/ procedure call with the following property: the call may use the input value but does 
not update the argument.
See Section~\ref{terms:semantic} on page~\pageref{terms:semantic}.

%%%%%%%%%%%%%%%%%%%%%%%%%%%%%%%%%%%%%%%%%%%%%%%%%%%%%%%%%%%%%%%%%%%%%%
\label{glossary:INOUT}
\item \bf{ INOUT} \\*
An argument of an \MPI/ procedure call with the following property: the call may both use and update the argument.
See Section~\ref{terms:semantic} on page~\pageref{terms:semantic}.

%%%%%%%%%%%%%%%%%%%%%%%%%%%%%%%%%%%%%%%%%%%%%%%%%%%%%%%%%%%%%%%%%%%%%%
\label{glossary:in_place}
\item \bf{ in place} \\*
A collective communication in which the output
buffer is identical to the input buffer.  This is specified by
providing a special argument value, \const{MPI\_IN\_PLACE}, instead of the
send buffer or the receive buffer argument,
depending on the operation performed. 
See Section~\ref{sec:coll-communicator} on page~\pageref{sec:coll-communicatorr}.

%%%%%%%%%%%%%%%%%%%%%%%%%%%%%%%%%%%%%%%%%%%%%%%%%%%%%%%%%%%%%%%%%%%%%%
\label{glossary:intercommunicator}
\item \bf{ intercommunicator} \\*
A communicator that identifies two distinct groups of processes
linked with a context.  (See also ~\ref{glossary:intracommunicator}.)
See Section~\ref{sec:coll-communicator} on page~\pageref{sec:coll-communicator}.

%%%%%%%%%%%%%%%%%%%%%%%%%%%%%%%%%%%%%%%%%%%%%%%%%%%%%%%%%%%%%%%%%%%%%%
\label{glossary:interface}
\item \bf{ interface} \\* 
Syntax and semantics for invoking services from within an executing application.  
See Section~\ref{terms:semantic} on page~\pageref{terms:semantic}.

%%%%%%%%%%%%%%%%%%%%%%%%%%%%%%%%%%%%%%%%%%%%%%%%%%%%%%%%%%%%%%%%%%%%%%
\label{glossary:internal}
\item \bf{ internal} \\* 
Data in \emph{internal} data representation can be used for I/O operations in a homogeneous or
heterogeneous environment; the implementation will perform type
conversions if necessary. The implementation is free to store data in
any format of its choice,
with the restriction that it will maintain constant extents
for all predefined datatypes in any one file.
The environment in which the resulting file 
can be reused is implementation-defined
and must be documented by the implementation.
(See also ~\ref{glossary:native} and ~\ref{glossary:external32}.)
See Section~\ref{sec:io-file-interop} on page~\pageref{sec:io-file-interop}.

%%%%%%%%%%%%%%%%%%%%%%%%%%%%%%%%%%%%%%%%%%%%%%%%%%%%%%%%%%%%%%%%%%%%%%
\label{glossary:intracommunicator}
\item \bf{ intracommunicator} \\*
A communicator that can be thought of as an i\MPIreplace{3.0}{109}{n}{}dentifier for a single group of processes
linked with a context.   (See also ~\ref{glossary:intercommunicator}.)
See Section~\ref{sec:coll-communicator} on page~\pageref{sec:coll-communicator}.

%%%%%%%%%%%%%%%%%%%%%%%%%%%%%%%%%%%%%%%%%%%%%%%%%%%%%%%%%%%%%%%%%%%%%%
\label{glossary:lower_bound}
\item \bf{ lower bound} \\*
The displacement of the lowest unit of store which is addressed by the datatype. 
In general, if
\begin{displaymath}
Typemap = \{ (type_0 , disp_0 ) , ... , (type_{n-1} , disp_{n-1}) \} ,
\end{displaymath}
then the {\bf lower bound} of $Typemap$ is defined to be
\[
lb(Typemap) = \left\{ \begin{array}{ll}
\min_j disp_j & \mbox{if no entry has basic type {\sf lb}} \\
\min_j \{ disp_j \ \mbox{such that}\ type_j = {\sf lb} \} & \mbox{otherwise}
\end{array}
\right. \]
(See also ~\ref{glossary:lower_bound} and ~\ref{glossary:upper_bound}.)
See Section~\ref{pt2pt-exX} on page~\pageref{pt2pt-exX}.

%%%%%%%%%%%%%%%%%%%%%%%%%%%%%%%%%%%%%%%%%%%%%%%%%%%%%%%%%%%%%%%%%%%%%%
\label{glossary:matching}
\item \bf{ matching} \\*
(a) A language type (e.g., float) matches an MPI datatype (e.g., \type{MPI\_FLOAT}). 
(b) Two datatypes match if their type signatures are identical.
(c) A message matches a receive operation, if the communicator is identical, and the source rank and tag match, i.e., are identical or wildcarding isused.
 See Section~\ref{subsec:pt2pt-typematch} on page~\pageref{subsec:pt2pt-typematch}.

%%%%%%%%%%%%%%%%%%%%%%%%%%%%%%%%%%%%%%%%%%%%%%%%%%%%%%%%%%%%%%%%%%%%%%
\label{glossary:message_envelope}
\item \bf{ message envelope} \\*
In addition to the data part, messages carry information that can be used to
distinguish messages and selectively receive them.  This information consists
of a fixed number of fields, which we collectively call
the {\bf message envelope}.   These fields are
\begin{center}
source \\
destination \\
tag \\
communicator
\end{center}
See Section~\ref{subsec:pt2pt-envelope} on page~\pageref{subsec:pt2pt-envelope}.

%%%%%%%%%%%%%%%%%%%%%%%%%%%%%%%%%%%%%%%%%%%%%%%%%%%%%%%%%%%%%%%%%%%%%%
\label{glossary:native}
\item \bf{ native} \\*
Data in \emph{native} representation is stored in a file exactly
as it is in memory.
The advantage of this data representation is that
data precision and I/O performance are not lost in type conversions
with a purely homogeneous environment.
The disadvantage is the loss of transparent interoperability within a
heterogeneous \MPI/ environment.
(See also ~\ref{glossary:internal} and ~\ref{glossary:external32}.)
See Section~\ref{sec:io-file-interop} on page~\pageref{sec:io-file-interop}.

%%%%%%%%%%%%%%%%%%%%%%%%%%%%%%%%%%%%%%%%%%%%%%%%%%%%%%%%%%%%%%%%%%%%%%
\label{glossary:non-local}
\item \bf{ non-local} \\*
A procedure is non-local if completion of the operation may require
the execution of some \MPI/ procedure on another process.  Such an
operation may require
communication occurring with another user process.
See Section~\ref{terms:semantic} on page~\pageref{terms:semantic}.

%%%%%%%%%%%%%%%%%%%%%%%%%%%%%%%%%%%%%%%%%%%%%%%%%%%%%%%%%%%%%%%%%%%%%%
\label{glossary:non-overtaking}
\item \bf{ non-overtaking} \\* 
The requirement that
if a sender sends two messages in succession to the same destination, and
both match the same receive, then this operation cannot receive the
second message if the first one is still pending.
If a receiver posts two receives in succession, and both match the same
message,
then the second receive operation cannot be satisfied by this message, if the
first one is still pending.
See Section~\ref{terms:semantic} on page~\pageref{terms:semantic}.

%%%%%%%%%%%%%%%%%%%%%%%%%%%%%%%%%%%%%%%%%%%%%%%%%%%%%%%%%%%%%%%%%%%%%%
\label{glossary:nonblocking}
\item \bf{ nonblocking} \\*
 A procedure is nonblocking if the procedure may return before the
operation completes, and before the user is allowed to reuse
resources (such as buffers) specified in the call.
A nonblocking request is {\bf started} by the call that initiates it, e.g.,
\mpifunc{MPI\_ISEND}.
The word complete is used with respect to operations, requests, and
communications.  An {\bf operation completes} when the user is allowed
to reuse resources, and any output buffers have been updated; i.e. a
call to \mpifunc{MPI\_TEST} will return
\mpiarg{flag}~=~\consti{true}.  A {\bf request is completed} by a call
to wait, which returns, or a test or get status call which returns
\mpiarg{flag}~=~\consti{true}.  This completing call has two effects:
the status is extracted from the request; in the case of test and
wait, if the request was nonpersistent, it is
% {\bf freed}.
{\bf freed}, and becomes {\bf inactive} if it was persistent.
A {\bf communication completes} when all participating operations complete.
See Section~\ref{sec:pt2pt-semantics} on page~\pageref{sec:pt2pt-semantics}.

%%%%%%%%%%%%%%%%%%%%%%%%%%%%%%%%%%%%%%%%%%%%%%%%%%%%%%%%%%%%%%%%%%%%%%
\label{glossary:nondeterminism}
\item \bf{ nondeterminism} \\*
A nondeterministic program is one in which either (a) repeated executions 
of the program with the same input may yield different results (weak 
nondeterminism); or (b) the sequence of states through which the program 
passes is not uniquely determined by the input even if the results are the 
same (strong nondeterminism). Nondeterminism may originate from the 
use of wildcards, \func{MPI\_CANCEL}, \func{MPI\_WAITANY}, rounding errors in floating-point 
reduction operations, and so on.
See Section~\ref{sec:ei-error} on page~\pageref{sec:ei-error}.

%%%%%%%%%%%%%%%%%%%%%%%%%%%%%%%%%%%%%%%%%%%%%%%%%%%%%%%%%%%%%%%%%%%%%%
\label{glossary:null_handle}
\item \bf{ null handle} \\*
A handle with value \linebreak \const{MPI\_REQUEST\_NULL}.
See Section~\ref{subsec:pt2pt-commend} on page~\pageref{subsec:pt2pt-commend}.


%%%%%%%%%%%%%%%%%%%%%%%%%%%%%%%%%%%%%%%%%%%%%%%%%%%%%%%%%%%%%%%%%%%%%%
\label{glossary:null_process}
\item \bf{ null process} \\*
A ``dummy'' source or
destination
for communication.  This simplifies the code that is needed for dealing with
boundaries, for example, in the case of a non-circular shift done with calls to
send-receive.
See Section~\ref{sec:pt2pt-nullproc} on page~\pageref{sec:pt2pt-nullproc}.

%%%%%%%%%%%%%%%%%%%%%%%%%%%%%%%%%%%%%%%%%%%%%%%%%%%%%%%%%%%%%%%%%%%%%%
\label{glossary:offset}
\item \bf{ offset} \\* 
An {\it offset} is a position
in the file
relative to the current view,
expressed as a count of etypes.
Holes in the view's filetype are skipped when calculating this position.
Offset 0 is the location of the first etype visible in the view
(after skipping the displacement and any initial holes in the view).
% The beginning of the view,
% which is {\it displacement} bytes from the beginning of the file,
% is at offset 0.
% Holes in the view's filetype are ignored when calculating this position.
For example, an offset of 2 for process 1
in Figure~\ref{fig:gloss-io-comp-filetypes} is the position
of the 8th etype in the file after the displacement.
An ``explicit offset'' is an offset that is used as a formal parameter
in explicit data access routines.
See Section~\ref{subsec:io-2:definitions} on page~\pageref{subsec:io-2:definitions}.

%%%%%%%%%%%%%%%%%%%%%%%%%%%%%%%%%%%%%%%%%%%%%%%%%%%%%%%%%%%%%%%%%%%%%%
\label{glossary:opaque}
\item \bf{ opaque} \\*
Data objects managed by MPI whose size and shape are not visible to the user. These objects are said to reside in system space.
Opaque objects are accessed by the application program through {\bf handles}, e.g., a communicator handles. 
See Section~\ref{terms:opaque-objects} on page~\pageref{terms:opaque-objects}.

%%%%%%%%%%%%%%%%%%%%%%%%%%%%%%%%%%%%%%%%%%%%%%%%%%%%%%%%%%%%%%%%%%%%%%
\label{glossary:order}
\item \bf{ order} \\*
The requirement that operations be performed as the sequence of the calls that initiate the communication.
See Section~\ref{subsec:pt2pt-semantics} on page~\pageref{subsec:pt2pt-semantics}.

%%%%%%%%%%%%%%%%%%%%%%%%%%%%%%%%%%%%%%%%%%%%%%%%%%%%%%%%%%%%%%%%%%%%%%
\label{glossary:origin}
\item \bf{ origin} \\*
The process that performs the call in on-sided communications.
See Section~\ref{sec:one-side-2} on page~\pageref{sec:one-side-2}.

%%%%%%%%%%%%%%%%%%%%%%%%%%%%%%%%%%%%%%%%%%%%%%%%%%%%%%%%%%%%%%%%%%%%%%
\label{glossary:OUT}
\item \bf{ OUT} \\*
An argument of an \MPI/ procedure call with the following property: the call may update the argument but does not use its input value.
See Section~\ref{terms:semantic} on page~\pageref{terms:semantic}.

%%%%%%%%%%%%%%%%%%%%%%%%%%%%%%%%%%%%%%%%%%%%%%%%%%%%%%%%%%%%%%%%%%%%%%
\label{glossary:pack}
\item \bf{ pack} \\*
The process of copying data into a contiguous buffer
before sending it. (See also ~\ref{glossary:unpack}.)
See Section~\ref{sec:pt2pt-packing} on page~\pageref{sec:pt2pt-packing}.

%%%%%%%%%%%%%%%%%%%%%%%%%%%%%%%%%%%%%%%%%%%%%%%%%%%%%%%%%%%%%%%%%%%%%%
\label{glossary:passive_target}
\item \bf{ passive target} \\*
An \RMA/ communication where data is moved from the memory of one
process to the memory of another, and only the origin process is
explicitly involved
in
the transfer.  Thus, two origin processes may communicate by accessing
the same location in a target window.  The process that owns the
target window may be distinct from the two communicating processes, 
in which case it does not participate explicitly in the communication.
This communication
paradigm is closest to a shared memory model, where shared data can be
accessed by all processes, irrespective 
of location. (See also ~\ref{glossary:active_target}.)
See Section~\ref{sec:1sided-sync} on page~\pageref{sec:1sided-sync}.

%%%%%%%%%%%%%%%%%%%%%%%%%%%%%%%%%%%%%%%%%%%%%%%%%%%%%%%%%%%%%%%%%%%%%%
\label{glossary:persistent_communication_request}
\item \bf{ persistent communication request} \\*
An optimization available when a communication with the same argument list is repeatedly
executed within the inner loop of a parallel computation.  In such a
situation, it may be possible to optimize the communication by
binding the list of communication arguments to a {\bf persistent} communication
request once and, then, repeatedly using
the request to initiate and complete messages.  The
persistent request thus created can be thought of as a
communication port or a ``half-channel.''
It does not provide the full functionality of a conventional channel,
since there is no binding of the send port to the receive port. This
construct allows reduction of the overhead for communication
between the process and communication controller, but not of the overhead for
communication between one communication controller and another.
It is not necessary that messages sent with a persistent request be received
by a receive operation using a persistent request, or vice versa..
See Section~\ref{sec:pt2pt-persistent} on page~\pageref{sec:pt2pt-persistent}.

%%%%%%%%%%%%%%%%%%%%%%%%%%%%%%%%%%%%%%%%%%%%%%%%%%%%%%%%%%%%%%%%%%%%%%
\label{glossary:PMPI}
\item \bf{ PMPI} \\*
Profiling MPI interface. A name-shift interface that provides a mechanism to analyze MPI function usage.
See Section~\ref{sec:c++profile} on page~\pageref{sec:c++profile}.

%%%%%%%%%%%%%%%%%%%%%%%%%%%%%%%%%%%%%%%%%%%%%%%%%%%%%%%%%%%%%%%%%%%%%%
\label{glossary:point-to-point}
\item \bf{ point-to-point} \\*
Messages delivered from one sending process to one receiving process. (See also ~\ref{glossary:collective}.)
See Section~\ref{sec:pt2pt-intro} on page~\pageref{sec:pt2pt-intro}.

%%%%%%%%%%%%%%%%%%%%%%%%%%%%%%%%%%%%%%%%%%%%%%%%%%%%%%%%%%%%%%%%%%%%%%
\label{glossary:port_name}
\item \bf{ port name} \\*
A \mpiarg{port\_name} is a {\em system-supplied} string that encodes a
low-level network address at which a server can be
contacted. Typically this is an IP address and a port number, but an
implementation is free to use any protocol.
See Section~\ref{sec:client-server} on page~\pageref{sec:client-server}.

%%%%%%%%%%%%%%%%%%%%%%%%%%%%%%%%%%%%%%%%%%%%%%%%%%%%%%%%%%%%%%%%%%%%%%
\label{glossary:portable}
\item \bf{ portable} \\*
A datatype is portable, if it is a predefined datatype, or it is derived
from a portable datatype using only the type constructors
\mpifunc{MPI\_TYPE\_CONTIGUOUS}, \mpifunc{MPI\_TYPE\_VECTOR},
\mpifunc{MPI\_TYPE\_INDEXED},
% \mpifunc{MPI\_TYPE\_INDEXED\_BLOCK}, %% This name does not exist
\mpifunc{MPI\_TYPE\_CREATE\_INDEXED\_BLOCK},
\mpifunc{MPI\_TYPE\_CREATE\_SUBARRAY}, \mpifunc{MPI\_TYPE\_DUP}, and
\mpifunc{MPI\_TYPE\_CREATE\_DARRAY}.
Such a datatype is portable because all displacements in the datatype
are in terms of extents of one predefined datatype.  Therefore, if such a
datatype fits a data layout in one memory, it will fit the
corresponding data layout in another memory, if the same declarations
were used, even if the two systems have different architectures.  On
the other hand, if a datatype was constructed using
\mpifunc{MPI\_TYPE\_CREATE\_HINDEXED}, \mpifunc{MPI\_TYPE\_CREATE\_HVECTOR} or
\mpifunc{MPI\_TYPE\_CREATE\_STRUCT}, then the datatype contains explicit byte
displacements (e.g., providing padding to meet alignment restrictions).
These displacements are unlikely to be chosen correctly if they fit
data layout on one memory, but are used for data layouts on another
process, running on a processor with a different architecture.
See Section~\ref{terms:semantic} on page~\pageref{terms:semantic}.

%%%%%%%%%%%%%%%%%%%%%%%%%%%%%%%%%%%%%%%%%%%%%%%%%%%%%%%%%%%%%%%%%%%%%%
\label{glossary:predefined_datatype}
\item \bf{ predefined datatype} \\*
A predefined datatype is a datatype with a predefined (constant) name
(such as \consti{MPI\_INT}, \consti{MPI\_FLOAT\_INT}, or \consti{MPI\_UB})
or a datatype constructed with \mpifunc{MPI\_TYPE\_CREATE\_F90\_INTEGER},
\mpifunc{MPI\_TYPE\_CREATE\_F90\_REAL}, or
\mpifunc{MPI\_TYPE\_CREATE\_F90\_COMPLEX}.  The former are {\bf named}
whereas the latter are {\bf unnamed}.
(See also ~\ref{glossary:general_datatype} and ~\ref{glossary:derived_datatype} .)
See Section~\ref{terms:semantic} on page~\pageref{terms:semantic}.

%%%%%%%%%%%%%%%%%%%%%%%%%%%%%%%%%%%%%%%%%%%%%%%%%%%%%%%%%%%%%%%%%%%%%%
\label{glossary:process}
\item \bf{ process} \\*
An \MPI/ program consists of autonomous processes, executing their own
code, in an 
MIMD style. The codes executed by each process need not be
identical.  The processes communicate via calls to \MPI/ communication
primitives.  Typically, each process executes in its own address
space, although shared-memory implementations of \MPI/ are possible.
A process is represented in \MPI/ by a (group, rank) pair. 
A (group, rank) pair specifies a unique process but 
a process does not determine a unique (group, rank) pair, since
a process may belong to several groups.
See Section~\ref{sec:macros} on page~\pageref{sec:macros}.

%%%%%%%%%%%%%%%%%%%%%%%%%%%%%%%%%%%%%%%%%%%%%%%%%%%%%%%%%%%%%%%%%%%%%%
\label{glossary:process_group}
\item \bf{ process group} \\*
An ordered list of processes that share a communicator context. Processes are identified by their
rank within this group.  Thus, the range of valid values for \mpiarg{dest} is
{\sf 0, ... , n-1}, where {\sf n} is the number of
processes in the group.  
See Section~\ref{subsec:pt2pt-envelope} on page~\pageref{subsec:pt2pt-envelope}.

%%%%%%%%%%%%%%%%%%%%%%%%%%%%%%%%%%%%%%%%%%%%%%%%%%%%%%%%%%%%%%%%%%%%%%
\label{glossary:progress}
\item \bf{ progress} \\*
The requirement that
if a pair of
matching send and receives have been initiated on two processes, then at
least one of these two operations will complete, independently of
other actions in the system:  the send operation will
complete, unless the receive is satisfied by another message, and
completes; the
receive operation will complete, unless the message sent is consumed by another
matching receive that was posted at the same destination process.
See Section~\ref{pt2pt-exE} on page~\pageref{pt2pt-exE}.

%%%%%%%%%%%%%%%%%%%%%%%%%%%%%%%%%%%%%%%%%%%%%%%%%%%%%%%%%%%%%%%%%%%%%%
\label{glossary:ready_communication_mode}
\item \bf{ ready communication mode} \\*
A communication protocol in which the communication
may be started {\em only} if the matching receive is already posted.
Otherwise, the operation is erroneous and its outcome is undefined.
On some systems, this allows the removal of a hand-shake
operation that is otherwise required and results in improved
performance.
The completion of the send operation does not depend on the
status of a matching receive, and merely indicates that the send
buffer can be reused.   A send operation that uses the ready mode has
the same semantics as a standard send operation, or a synchronous send
operation; it is merely that the sender provides additional
information to the system (namely that a matching receive is already
posted), that can save some overhead.  In a correct program, therefore, a
ready send could be replaced by a standard send with no effect on the
behavior of the program other than performance.
(See also ~\ref{glossary:standard_communication_mode}, 
~\ref{glossary:buffered_communication_mode},
~\ref{glossary:synchronous_communication_mode}.)
See Section~\ref{sec:pt2pt-modes} on page~\pageref{sec:pt2pt-modes}.


%%%%%%%%%%%%%%%%%%%%%%%%%%%%%%%%%%%%%%%%%%%%%%%%%%%%%%%%%%%%%%%%%%%%%%
\label{glossary:reduce}
\item \bf{ reduce} \\*
A collective operation that determines a result for all members of a group.
The reduction operation can be either one of a predefined list of
operations, or a user-defined operation.
The global reduction functions come in several flavors: a reduce that
returns the result of the reduction
%at one node,
to one member of a group,
an all-reduce that
returns this result
%at all nodes,
to all members of a group,
and 
% a scan (parallel prefix) operation.  
two 
scan (parallel prefix) 
operations.  
See Section~\ref{global-reduce} on page~\pageref{global-reduce}.

%%%%%%%%%%%%%%%%%%%%%%%%%%%%%%%%%%%%%%%%%%%%%%%%%%%%%%%%%%%%%%%%%%%%%%
\label{glossary:representation_conversion}
\item \bf{ type conversion} \\*
Changing the binary representation of a value,
e.g., from Hex floating point to IEEE floating point. Note that the buffer size required for the receive can be affected by data conversions and
by the stride of the receive datatype. No conversion need occur when an \MPI/ program executes in
a homogeneous system, where all processes run in the same environment. (See also ~\ref{glossary:representation_conversion}.)
See Section~\ref{subsec:pt2pt-conversion} on page~\pageref{subsec:pt2pt-conversion}.

%%%%%%%%%%%%%%%%%%%%%%%%%%%%%%%%%%%%%%%%%%%%%%%%%%%%%%%%%%%%%%%%%%%%%%
\label{glossary:RMA}
\item \bf{ RMA} \\*
Remote Memory Access.
In some systems, message-passing and remote-memory-access (\RMA/) operations
run faster when accessing specially allocated memory (e.g., memory that is
shared by the other processes in the communicating group on an SMP).  \MPI/
provides a mechanism for allocating and freeing such special memory.  The use
of such memory for message-passing or \RMA/ is not mandatory, and this memory
can be used without restrictions as any other dynamically allocated memory.
(See also ~\ref{glossary:active_target} and ~\ref{glossary:passive_target}.)
See Section~\ref{sec:misc-memalloc} on page~\pageref{sec:misc-memalloc}.

%%%%%%%%%%%%%%%%%%%%%%%%%%%%%%%%%%%%%%%%%%%%%%%%%%%%%%%%%%%%%%%%%%%%%%
\label{glossary:root}
\item \bf{ root} \\*
The single process that originates or receives communication for those collective
operations that originate or receive to one process (e.g., broadcast and gather). 
See Section~\ref{sec:coll-intro} on page~\pageref{sec:coll-intro}.

%%%%%%%%%%%%%%%%%%%%%%%%%%%%%%%%%%%%%%%%%%%%%%%%%%%%%%%%%%%%%%%%%%%%%%
\label{glossary:scope}
\item \bf{ scope} \\*
The domain over which the \mpiarg{service\_name} 
can be retrieved.
[See Section~\ref{sec:client-server} on page~\pageref{sec:client-server}.

%%%%%%%%%%%%%%%%%%%%%%%%%%%%%%%%%%%%%%%%%%%%%%%%%%%%%%%%%%%%%%%%%%%%%%
\label{glossary:send-receive_operation}
\item \bf{ send-receive operation} \\*
Operations that combine in one call the sending of a
message to one destination and the receiving of another message, from
another process.  The two (source and destination) are possibly the same.
A send-receive operation is
very useful for executing a shift operation across a chain of
processes.  If blocking sends and receives are used for such a shift,
then one needs to order the sends and receives correctly (for
example, even processes
send, then receive, odd processes receive first, then send) so as to prevent
cyclic dependencies that may lead to deadlock.  When a send-receive
operation is used, the communication subsystem takes care of
these issues.
See Section~\ref{sec:pt2pt-sendrecv} on page~\pageref{sec:pt2pt-sendrecv}.

%%%%%%%%%%%%%%%%%%%%%%%%%%%%%%%%%%%%%%%%%%%%%%%%%%%%%%%%%%%%%%%%%%%%%%
\label{glossary:sequential_storage}
\item \bf{ sequential storage} \\*
Variables that belong to the same array,
to the same {\sf COMMON} block in Fortran, or to the same structure in C.
See Section~\ref{subsec:pt2pt-segmented} on page~\pageref{subsec:pt2pt-segmented}.

%%%%%%%%%%%%%%%%%%%%%%%%%%%%%%%%%%%%%%%%%%%%%%%%%%%%%%%%%%%%%%%%%%%%%%
\label{glossary:serialized}
\item \bf{ sequential storage} \\* 
MPI calls are not made concurrently from two distinct threads (all MPI calls are \emph{serialized}.
See Section~\ref{sec:ei-threads} on page~\pageref{sec:ei-threads}.


%%%%%%%%%%%%%%%%%%%%%%%%%%%%%%%%%%%%%%%%%%%%%%%%%%%%%%%%%%%%%%%%%%%%%%
\label{glossary:server}
\item \bf{ server} \\*
 \MPI/ provides a mechanism for two sets of \MPI/  processes that do not share a communicator
to establish communication.
Establishing contact between two groups of processes that do not share an
existing communicator is a collective but asymmetric process.  One group of
processes indicates its willingness to accept connections from other groups of
processes.  We will call this group the (parallel) \emph{server}, even if this
is not a client/server type of application.  
(See also ~\ref{glossary:client}). 
See Section~\ref{sec:client-server} on page~\pageref{sec:client-server}.

%%%%%%%%%%%%%%%%%%%%%%%%%%%%%%%%%%%%%%%%%%%%%%%%%%%%%%%%%%%%%%%%%%%%%%
\label{glossary:split_collective}
\item \bf{ split collective} \\* 
A restricted form of ``nonblocking'' operations
for collective file data access.
These routines are referred to as ``split'' collective routines
because a single collective operation is split in two:
a begin routine and an end routine.
The begin routine begins the operation,
much like a nonblocking data access (e.g., \func{MPI\_FILE\_IREAD}).
The end routine completes the operation,
much like the matching test or wait (e.g., \func{MPI\_WAIT}).
As with nonblocking data access operations,
the user must not use the buffer
% the user must not change the parameters
passed to a begin routine while the routine is outstanding;
the operation must be completed with an end routine before it
is safe to free buffers, etc.
See Section~\ref{sec:io-split-collective} on page~\pageref{sec:io-split-collective}.

%%%%%%%%%%%%%%%%%%%%%%%%%%%%%%%%%%%%%%%%%%%%%%%%%%%%%%%%%%%%%%%%%%%%%%
\label{glossary:standard_communication_mode}
\item \bf{ standard communication mode} \\*
A communication protocol that leaves it up to \MPI/ to decide whether outgoing
messages will be buffered.  \MPI/ may
buffer outgoing messages.  In such a case, the send call may complete
before a matching receive is invoked.  On the other hand, buffer space may be
unavailable, or \MPI/ may choose not to buffer
outgoing messages, for performance reasons. In this case,
the send call will not complete until a matching receive has been posted, and
the data has been moved to the receiver.
Thus, a send in standard mode can be started whether or not a
matching receive has been posted.  It may complete before a matching receive
is posted.  The
standard mode send is {\bf non-local}:  successful completion of the send
operation may depend on the occurrence of a matching receive. 
(See also ~\ref{glossary:buffered_communication_mode}, 
~\ref{glossary:synchronous_communication_mode},
~\ref{glossary:ready_communication_mode}.)
See Section~\ref{sec:pt2pt-modes} on page~\pageref{sec:pt2pt-modes}.

%%%%%%%%%%%%%%%%%%%%%%%%%%%%%%%%%%%%%%%%%%%%%%%%%%%%%%%%%%%%%%%%%%%%%%
\label{glossary:synchronizing}
\item \bf{ synchronizing} \\*
Communication between \MPI/ processes with the effect of constraining the relative 
order that those \MPI/  processes execute code. For example, \func{MPI\_BARRIER} 
blocks the caller until all group members have called it. The call returns at any process 
only after all group members have
entered the call.
See Section~\ref{sec:coll-intro} on page~\pageref{sec:coll-intro}.

%%%%%%%%%%%%%%%%%%%%%%%%%%%%%%%%%%%%%%%%%%%%%%%%%%%%%%%%%%%%%%%%%%%%%%
\label{glossary:synchronous_communication_mode}
\item \bf{ synchronous communication mode} \\*
A communication protocol in which the
communication can be started whether or
not a matching receive was posted.  However, the send will complete
successfully only if a matching receive is posted, and the
receive operation has started to receive the message sent by the
synchronous send.
Thus, the completion of a synchronous send not only indicates that the
send buffer can be reused, but 
it also indicates that the receiver has
reached a certain point in its execution, namely that it has started
executing the matching receive.  If both sends and receives are
blocking operations then the use of the synchronous mode provides
synchronous communication semantics: a communication does not complete
at either end before both processes rendezvous at the
communication.  A send executed in this mode is non-local.
(See also ~\ref{glossary:standard_communication_mode}, 
~\ref{glossary:buffered_communication_mode},
~\ref{glossary:ready_communication_mode}.)
See Section~\ref{sec:pt2pt-modes} on page~\pageref{sec:pt2pt-modes}.

%%%%%%%%%%%%%%%%%%%%%%%%%%%%%%%%%%%%%%%%%%%%%%%%%%%%%%%%%%%%%%%%%%%%%%
\label{glossary:target}
\item \bf{ target} \\*
The process in which the memory is accessed in on-sided communications.
See Section~\ref{sec:one-side-2} on page~\pageref{sec:one-side-2}.

%%%%%%%%%%%%%%%%%%%%%%%%%%%%%%%%%%%%%%%%%%%%%%%%%%%%%%%%%%%%%%%%%%%%%%
\label{glossary:thread_safe}
\item \bf{ thread safe} \\*
The property that two concurrently
running threads may make \MPI/ calls and the outcome will be as if the
calls executed in some order, even if their execution is interleaved.
See Section~\ref{sec:coll-intro} on page~\pageref{sec:coll-intro}.

%%%%%%%%%%%%%%%%%%%%%%%%%%%%%%%%%%%%%%%%%%%%%%%%%%%%%%%%%%%%%%%%%%%%%%
\label{glossary:topology}
\item \bf{ topology} \\* 
A topology is an extra,
optional attribute that one can give to an intra-communicator; topologies
cannot be added to inter-communicators.  A topology can provide a convenient
naming mechanism for the processes of a group (within a communicator), and
additionally, may assist the runtime system in mapping the processes onto
hardware.
(See also ~\ref{glossary:groups} .)
See Section~\ref{sec:topol} on page~\pageref{sec:topol}.

%%%%%%%%%%%%%%%%%%%%%%%%%%%%%%%%%%%%%%%%%%%%%%%%%%%%%%%%%%%%%%%%%%%%%%
\label{glossary:true_extent}
\item \bf{ true extent} \\*
The true size of a datatype, i.e., the extent of the corresponding typemap, ignoring
\consti{MPI\_LB} and \mpiarg{MPI\_UB} markers, and performing no
rounding for alignment.  The \mpiarg{true\_extent} is the minimum number of bytes of
memory necessary to hold a datatype, uncompressed. (Note that this applies
to situations such as spanning trees; since the receive buffer is only valid on the
root process, one will need to allocate some temporary space for
receiving data on intermediate nodes.  However, the datatype extent
cannot be used as an estimate of the amount of space that needs to be
allocated, if the user has modified the extent
using the \consti{MPI\_UB} and \consti{MPI\_LB} values.)
If the typemap associated with
\mpiarg{datatype} is
\[
Typemap = \{ (type_0, disp_0), \ldots , (type_{n-1}, disp_{n-1})\}
\]
Then
\[
true\_lb(Typemap) = min_j  \{ disp_j ~:~ type_j \ne {\bf lb, ub} \},
\]
\[
true\_ub (Typemap) = max_j \{disp_j + sizeof(type_j) ~:~ type_j \ne
{\bf lb, ub}\} ,
\]
and
\[
true\_extent (Typemap) = true\_ub(Typemap) - true\_lb(typemap).
\]
(See also ~\ref{glossary:lower_bound}, ~\ref{glossary:upper_bound}, and ~\ref{glossary:extent}.)
See Section~\ref{sec:pt2pt-datatype} on page~\pageref{sec:pt2pt-datatype}.

%%%%%%%%%%%%%%%%%%%%%%%%%%%%%%%%%%%%%%%%%%%%%%%%%%%%%%%%%%%%%%%%%%%%%%
\label{glossary:type_conversion}
\item \bf{ type conversion} \\*
Changing the datatype of a value, e.g., by rounding a
\ftype{REAL} to an \ftype{INTEGER}. Note that the buffer size required for the receive can be affected by data conversions and
by the stride of the receive datatype. No conversion need occur when an \MPI/ program executes in
a homogeneous system, where all processes run in the same environment. (See also ~\ref{glossary:representation_conversion}.)
See Section~\ref{subsec:pt2pt-conversion} on page~\pageref{subsec:pt2pt-conversion}.


%%%%%%%%%%%%%%%%%%%%%%%%%%%%%%%%%%%%%%%%%%%%%%%%%%%%%%%%%%%%%%%%%%%%%%
\label{glossary:type_map}
\item \bf{ type map} \\*
The pair of sequences (or sequence of pairs) associated with a general dataype.
The displacements are not required to be positive, distinct, or
in increasing order. Therefore, the order of items need not
coincide with their order in store, and an item may appear more than
once. Type maps take the form
\[
Typemap = \{ (type_0,disp_0), ..., (type_{n-1}, disp_{n-1}) \} ,
\]
be such a type map, where $type_i$ are basic types, and
$disp_i$ are  displacements.
(See also ~\ref{glossary:type_signature}.)
See Section~\ref{sec:pt2pt-datatype} on page~\pageref{sec:pt2pt-datatype}.

%%%%%%%%%%%%%%%%%%%%%%%%%%%%%%%%%%%%%%%%%%%%%%%%%%%%%%%%%%%%%%%%%%%%%%
\label{glossary:type_signature}
\item \bf{ type signature} \\*
The sequences of types associated with a general dataype.
Type signatures may be used to validate matching types between sender and receiver; they take the form
\[
Typesig = \{ type_0 , ... , type_{n-1} \}
\]
where $type_i$ are basic types.
(See also ~\ref{glossary:type_map}.)
See Section~\ref{sec:pt2pt-datatype} on page~\pageref{sec:pt2pt-datatype}.

%%%%%%%%%%%%%%%%%%%%%%%%%%%%%%%%%%%%%%%%%%%%%%%%%%%%%%%%%%%%%%%%%%%%%%
\label{glossary:unpack}
\item \bf{ unpack} \\*
The process of copying data into a contiguous buffer
after receiving it. (See also ~\ref{glossary:pack}.)
See Section~\ref{sec:pt2pt-packing} on page~\pageref{sec:pt2pt-packing}.

%%%%%%%%%%%%%%%%%%%%%%%%%%%%%%%%%%%%%%%%%%%%%%%%%%%%%%%%%%%%%%%%%%%%%%
\label{glossary:upper_bound}
\item \bf{ upper bound} \\*
The displacement of the highest unit of store which is addressed by the datatype. 
In general, if
\begin{displaymath}
Typemap = \{ (type_0 , disp_0 ) , ... , (type_{n-1} , disp_{n-1}) \} ,
\end{displaymath}
then the {\bf upper bound} of $Typemap$ is defined to be
\[
ub(Typemap) = \left\{ \begin{array}{ll}
\max_j disp_j + sizeof(type_j) + \epsilon & \mbox{if no entry has basic type
{\sf ub}}
\\ \max_j \{ disp_j \ \mbox{such that}\ type_j = {\sf ub} \} & \mbox{otherwise}
\end{array}
\right. \]
(See also ~\ref{glossary:lower_bound} and ~\ref{glossary:lower_bound}.)
See Section~\ref{pt2pt-exX} on page~\pageref{pt2pt-exX}.

%%%%%%%%%%%%%%%%%%%%%%%%%%%%%%%%%%%%%%%%%%%%%%%%%%%%%%%%%%%%%%%%%%%%%%
\label{glossary:view}
\item \bf{ view} \\*
A {\it view} defines the current set of data visible
and accessible from an open file as an ordered set of etypes.
Each process has its own view of the file,
defined by three quantities:
a displacement, an etype, and a filetype.
The pattern described by a filetype is repeated,
beginning at the displacement, to define the view.
The pattern of repetition is defined to be the same pattern
that \func{MPI\_TYPE\_CONTIGUOUS} would produce if it were passed
the filetype and an arbitrarily large count.
Figure~\ref{fig:gloss-io-filetype} shows how the tiling works; note 
that the filetype in this example must have explicit 
lower and upper bounds set in order for the initial and final holes to be
repeated in the view.
Views can be changed by the user during program execution.
The default view is a linear byte stream
(displacement is zero, etype and filetype equal to \type{MPI\_BYTE}).

\begin{figure}[htpb]
  \centerline{\includegraphics[width=4.0in]{figures/io-filetype}}
  \caption{Etypes and filetypes}
  \label{fig:glossary-io-filetype}
\end{figure}

A group of processes can use complementary views to
achieve a global data distribution such as a scatter/gather pattern
(see Figure~\ref{fig:gloss-io-comp-filetypes}).

\begin{figure}[htpb]
  \centerline{\includegraphics[width=4.0in]{figures/io-comp-filetypes}}
  \caption{Partitioning a file among parallel processes}
  \label{fig:gloss-io-comp-filetypes}
\end{figure}
See Section~\ref{subsec:io-2:definitions} on page~\pageref{subsec:io-2:definitions}.

%%%%%%%%%%%%%%%%%%%%%%%%%%%%%%%%%%%%%%%%%%%%%%%%%%%%%%%%%%%%%%%%%%%%%%
\label{glossary:wildcard}
\item \bf{ wildcard} \\*
A special tag that will match all messages. Wildcard values may be used 
to accept all message \emph{sources} and/or \emph{tags}, but may not be used to 
constrain \emph{communicators}.
See Section~\ref{sec:one-side-2} on page~\pageref{sec:one-side-2}.

%%%%%%%%%%%%%%%%%%%%%%%%%%%%%%%%%%%%%%%%%%%%%%%%%%%%%%%%%%%%%%%%%%%%%%
\label{glossary:window}
\item \bf{ window} \\*
A range of memory that is made accessible to accesses by remote 
processes. In one-sided communications, each process specifies
a window of existing memory that it exposes to \RMA/ accesses by the
processes in the group of  \mpiarg{comm}.
The window consists of \mpiarg{size} bytes,
starting at address \mpiarg{base}.
See Section~\ref{sec:pt2pt-semantics} on page~\pageref{sec:pt2pt-semantics}.


\end{itemize}